\documentclass[a4paper,12pt]{article}


\usepackage{graphicx}


\usepackage{cmap}
\usepackage[T2A]{fontenc}
\usepackage[utf8]{inputenc}
\usepackage[english,russian]{babel}

\author{Никита Валов}
\title{Работа 3.2.4 - Свободные колебания в электрическом контуре}
\date{\today}


\begin{document}
	\maketitle
	
	\textbf{Цель работы:} исследование свободных колебаний в электрическом контуре
	
	
	\textbf{Оборудование:} генератор импульсов, электронная реле, магазин сопротивлений, магазин ёмкостей, катушка индуктивности, электронный осциллограф, универсальный измерительный мост.
	
	\textbf{Установка:}
	
	Для периодического возбуждения колебаний в контуре используется генератор импульсов Г5-54. С выхода генератора по коаксиальному кабелю импульсы поступают на колебателньый контур через электрнное реле, смонтированное в отдельном блоке(или на выходе генератора) Реле содержит диодный тиристор D и ограничительный резистор $R_1$.
	 Импульсы заряжают конденсатор C. После каждого импульса генератор отключается от колебательного контура, и в контуре возникают свободные затухающие колебания. Входное сопротивление осциллографа велико ($\approx1$ МОм), так что его влиянием на контур можно пренебречь. Для получения устойчиво картины затухающих колебаний используется режим ждущей развертки с синхронизацией внешними импульсами, поступающими с выхода "синхроимпульсы" генератора.
	
	
	\includegraphics[scale=0.5]{Inst}
	
	\textbf{Метод:}
	
	Схема изучения свободных колебаний:
	
	
	\includegraphics[scale=1.1]{teor1}
	
	
	
	\includegraphics[scale=1.1]{form}
	
	
	Исследуемый колебательный контур состоит из индуктивности L, ёмкости C и резистора R. Конденсатор контура заряжается короткими одиночными импульсами, после каждого из которых в кон-
туре возникают свободные затухающие колебания. Подав напряжение
с конденсатора на осциллограф, можно по картине, возникающей на
экране осциллографа, определить период колебаний в контуре, иссле-
довать затухание колебаний и определить основные параметры колеба-
тельного контура.

Картину колебаний можно представить на фазовой
плоскости. В этих координатах кривая незатухающих колебаний 
имеет вид эллипса (или окружности — при одинаковых амплитудах, а картина реальных колебаний изображается сворачивающейся
спиралью.

	\section{Задание}

	 Снимем с помощью осцилографа длину волны и количество колебаний.
	 Измерим период колебаний контура при разных ёмкостьях с помощью формулы:
	 
	 \begin{equation}
	 T_{exp}=\frac{T_0x}{nx_0}
	 \end{equation}
	 , где $T_0$ = 0.01с длина периода; х - расстояние которое занимает n периодов, $x_0$ расстояние между импульсами на экране при C = 0,02мкФ
	 Тогда мы найдем время одного колебания. 
	 
	 Погрешность измеренных данных будет составлять:
	 \begin{equation}
	 \varepsilon(T_{exp})^2	= \varepsilon(n)^2 + \varepsilon(x)^2 + \varepsilon(T_0)^2 + \varepsilon(x_0)^2
	 \end{equation}
	 
	 Таким образом получим эксперементальную зависимость $T_{exp}(C)$.
	 
	 Также мы можем расчитать период колебаний с помощью формулы
	 
	 \begin{equation}
	 T_{teor} = 2 \pi \sqrt{LC}
	 \end{equation}
	 
	 Погрешность измеренных данных будет составлять:
	 
	 \begin{equation}
	 \varepsilon(T)^2 = \frac{1}2(\varepsilon(L)^2+\varepsilon(C)^2)
	 \end{equation}
	 
	 Таким образом получим теоретическую зависимость $T_{teor}(C)$
	 
	 Теперь можем построить график $T_{exp} (T_{teor})$
	 
	 \includegraphics[scale=0.5]{graf}
	 
	 
	 На графике видно линейную зваисимость с погрешностью, которую дает измерение числа полуволн.
	 
	 \section{Задание:}
	 
	 Теперь расчитаем критическое сопротивление по формуле:
	 
	 \begin{equation}
	 R_{kr}=2\sqrt{\frac{L}C}
	 \end{equation}
	 
	 а С посчитаем при $w$ = 5кГЦ по формуле
	 \begin{equation}
	 w =\frac{1}{\sqrt{LC}}
	 \end{equation}
	 
	 	Погрешность считается по формуле:
	 	
	 	\begin{equation}
	 	\varepsilon(R)^2 = \frac{1}2(\varepsilon(L)^2+\varepsilon(C)^2)
	 	\end{equation}
	 	
	  при $L = 200\pm 2$ мГн получим $С = 5,05\pm0,05$ нФ и $R_{kr} = 12560\pm 130$ Ом
	  
	  
	  Теперь найдем критическое сопротивление эксперементально, добившись апереодичности колебаний. Полученное сопротивление $R_{kr}=10000$Ом
	  
	  
	  Далее для сопротивлений в диапазоне 0,1-0,3 от $R_{kr}$ будем считать коэффицент $\theta$ по формуле:
	  
	  \begin{equation}
	  \theta=\frac{1}n \ln{\frac{U_k}{U_{k+n}}}
	  \end{equation}
	  
	  Погрешность данного коэффицента считается:
	  
	  \begin{equation}
	  \sigma(\theta)^2 = \frac{\varepsilon(U_k)^2}{n^2} + \frac{\varepsilon(U_{k+n})^2}{n^2}
	  \end{equation}
	  
	  Все измеренные данные занесем в таблицу:
	  
	  \includegraphics{dat0}
	  
	  \includegraphics[scale=1.1]{dat1}
	
	Столбик $R,sum$ обозначает суммарное сопротивление вместе с оммическим сопротивление катушки и измеряется в кОм
	
	
	Теперь построим нужный график $\frac{1}{\theta^2}$ от $(\frac{1}{R^2})$ где R - суммарное сопротивление
	
	\includegraphics[scale=0.5]{graf2}
	
	
	
	Из формулы:
	
	\begin{equation}
	\frac{\pi}\theta=\frac{1}2\sqrt{\frac{R_{kr}^2}{R^2}-1}
	\end{equation}
	
	Можно увидеть, что критическое сопротивление находится с помощью коэффицента наклона k по формуле:

	
	\begin{equation}
	R_{kr} = 2\pi\sqrt{k}
	\end{equation}
	
	Расчитаем погрешности каждого из способов:
	
	Для способа теоретического:
	\begin{equation}
	\varepsilon(R_{kr}) = \frac{1}2(\varepsilon(L) + \varepsilon(C))
	\end{equation}
	
	Для способа графического:
	Расчитаем погрешность с помощью метода МНК и получим:
	
	$\varepsilon(K) \approx 8\%$
	
	Для эксперементального способа погрешность оценить тяжело, но в наших измерениях апереодичность наступала с точностью не более 1кОм
	
	Тогда конечный результат:
	

	Из графика:
	$R_{kr} = 11700 \pm 940$ Ом 
	
	Из расчета:
	$R_{kr} =12600 \pm 120$ Ом
	
	Из эскперемента:
	$R_{kr}= 10000 \pm 1000$ Ом
	
	Теоретический метод в пределах погрешности совпал с эскперементальным методом, поэтому можно считать, что теория подтверждена достаточно с хорошей точностью
	
	\section{Задание}
	Теперь проведем расчет добротноси контура Q по формуле:
	
	\begin{equation}
	Q_\theta=\frac{\pi}\theta
	\end{equation}
	
	относительная погрешность добротности равна декременту,
	и по второй формуле
	
	\begin{equation}
	Q_{Theory} = \frac{1}{R}\sqrt{\frac{L}{C}}
	\end{equation}
	погрешность равна:
	
	\begin{equation}
	\varepsilon(Q_{Theory})^2 = \varepsilon(R)^2 + \varepsilon(L)^2/2 + \varepsilon(C)^2/2
	\end{equation}
	
	Таблица значений, где Q1 посчитано с помощью $\theta$, а Q2 теоретическое значение.
	
	\includegraphics[scale=1.4]{dat2}
	
	При наибольшем декременте, получим $Q_1 = 5,47\pm 0,01$ и $Q_2 = 6,20\pm 0,12$
	
	При наименьшем декременте, получим $Q_1 = 2,06\pm 0,03$ и $Q_2 = 2,09\pm 0,04$
	
	\section{Задание}
	
	Теперь расчитаем декремент затухания с помощью спирали:
	
	\includegraphics[scale=1.3]{dat3}
	
	столбик $\theta(2)$ был расчитан с помощью спирали по формуле:
	
	\begin{equation}
	\theta=\frac{1}n \ln{\frac{R_k}{R_{k+n}}}
	\end{equation}
	
	его значения достаточно точно совпадает с $\theta$ посчитанным обыным методом
	
	
	
	\section{Вывод:}
	
	Полученные данные:
	
	\textbf{Параметры катушки:}
	
	$L = 200 \pm 1$ мГн
	$R = 30 \pm 10$ Ом ( при разных частотах, разное сопротивление)
	
	\textbf{Критическое сопротивление:}
	
	Из графика:
	$R_{kr} = 11700 \pm 940$ Ом 
	
	Из расчета:
	$R_{kr} =12600 \pm 120$ Ом
	
	Из эскперемента:
	$R_{kr}= 10000 \pm 1000$ Ом
	
	
	\textbf{Добротность:}

	
	При $\theta$ = 0,57 $Q_{exp} = 5,47 \pm 0,01$ $Q_{theor} = 6,20 \pm 0,12$
	
	При $\theta = 1,52$ $Q_{exp} = 2,06 \pm 0,03$ $Q_{theor} = 2,09 \pm 0,04$
	
	Декремент затухания при спиральном и обычном методе сходятся с довольно хорошей точностью:
	
	\includegraphics[scale=1.2]{thet}
	
	 таким образом эти методы оба подходят для вычисления декремента.
	
\end{document}