\documentclass[a4paper, 14pt]{extarticle}%тип документа

\date{}

\usepackage{graphicx}
\usepackage{cmap}
\usepackage[T2A]{fontenc}
\usepackage[utf8]{inputenc}
\usepackage{indentfirst}


\usepackage[english,russian]{babel}
\usepackage{multirow} % Слияние строк в таблице
\newcommand
{\un}[1]
{\ensuremath{\text{#1}}}

%Русский язык
\usepackage[T2A]{fontenc} %кодировка
\usepackage[utf8]{inputenc} %кодировка исходного кода
\usepackage[english,russian]{babel} %локализация и переносы

%Таблицы
\usepackage[table,xcdraw]{xcolor}
\usepackage{booktabs}

%Математика
\usepackage{amsmath, amsfonts, amssymb, amsthm, mathtools}

%отступы 
\usepackage[left=2cm,right=2cm,top=2cm,bottom=3cm,bindingoffset=0cm]{geometry}

%Графики
\usepackage{pgfplots}
\pgfplotsset{compat=1.9}

%Вставка картинок
\usepackage{graphicx}
\usepackage{wrapfig, caption}
\graphicspath{}
\DeclareGraphicsExtensions{.pdf,.png,.jpg, .jpeg}
\newcommand\ECaption[1]{%
     \captionsetup{font=footnotesize}%
     \caption{#1}}

\begin{document}

	% НАЧАЛО ТИТУЛЬНОГО ЛИСТА

\begin{titlepage}
	\begin{center}
		\large 	Московский физико-технический университет \\
		Физтех-школа радиотехники и компьютерных технологий\\
		\vspace{0.2cm}
		
		\vspace{4.5cm}
		Лабораторная работа № 1.2.3 \\ \vspace{0.2cm}
		\LARGE \textbf{Измерение магнитного поля Земли}
	\end{center}
	\vspace{2.3cm} \large
	
	\begin{center}
		Работу выполнил: \\
		Шурыгин Антон \\
		Б01-909

	\end{center}
	
	\begin{center} \vspace{60mm}
		г. Долгопрудный \\
	\end{center}
\end{titlepage}


	\textbf{Цель работы:} c помощью сцинтилляционного счетки измерить линейные коэффициенты ослабления потока $\gamma$-лучей в свинце, железе и алюминии; по их велечине определить энергию $\gamma$-квантов.

	
\section{Введение и краткая теория}

Интерферометр Майкельсона находит применение в спектрометрах с высоким разрешением, для абсолютных и относительных измерений длин с точностью 0,005 мкм. 

Оптическая схема интерферометра приведена на рис. 1. Источником света служит лазер $ЛГ$. Лазер излучает узкий пучок света, который фокусируется линзой Л1. В фокусе этой линзы возникает точечный
источник света S. Сферическая световая волна от источника S падает на делительный кубик ДК и делится его диагональной гранью на 
две волны — отражённую $1$ и проходящую $2$. Волна 1 отражается от
зеркала $З_1$, возвращается к кубику, частично проходит сквозь него и
попадает на экран $Э$. Волна $2$ отражается от зеркала $З_2$, частично отражается от кубика и также попадает на экран. Световые волны $1$ и $2$
испускаются одним источником S, и они когерентны между собой. Эти
волны создают на экране $Э$ интерференционную картину. Для увеличения масштаба интерференционной картины может быть использована
линза $Л_2$.

Зеркало З1 установлено перпендикулярно падающему лучу. Оно может перемещаться вдоль луча. Это зеркало в дальнейшем будет называться подвижным. Зеркало З2 вдоль направления падающего луча не
перемещается. Его, однако, можно наклонять по отношению к лучу.

\begin{figure}[h!]
    \centering
    \includegraphics[width=1.2\linewidth]{pics/scheme.png}
    \caption{Схема интерферометра}
    \label{}
\end{figure}




	\section{Экспериментальная установка}
	Схема установки, используемой в работе, показана на рис. \ref{scheme}. Свинцовый коллиматор выделяет узкий почти параллельный пучок $\gamma$-квантов, проходящий через набор поглотителей П и регистрируемый сцинтилляционным счетчиком). Сигналы от счетчика усиливаются и регистрируются пересчетным прибором ПП. Высоковольтный выпрямитель ВВ обеспечивает питание сцинтилляционного счетчика.

При недостаточно хорошей геометрии в результаты опытов могут
вкрасться существенные погрешности. В реальных установках всегда имеется конечная вероятность того, что $\gamma$-квант провзаимодействует в
поглотителе несколько раз до того, как попадет в детектор. Чтобы уменьшить число таких случаев, в данной работе сцинтилляционный счетчик расположен на большом расстоянии от источника $\gamma$-квантов, а поглотители имеют небольшие
размеры. Их следует устанавливать за коллиматорной щелью на некотором расстоянии друг от друга, чтобы испытавшие комптоновское
рассеяние и выбывшие из прямого потока кванты с меньшей вероятностью могли в него вернуться.
	
	\begin{figure}[h!]
		\centering
		\includegraphics[width=0.8\linewidth]{pics/scheme.png}
		\caption{Блок-схема установки, используемой для измерения коэффициентов ослабления потока $\gamma$-лучей.}
		\label{scheme}
	\end{figure} 

	\begin{figure}[h!]
		\centering
		\includegraphics[width=0.8\linewidth]{pics/part_disp.png}
		\caption{Схема рассеяния $\gamma$-квантов в поглотителе.}
		\label{}
	\end{figure}  

	\newpage
	
	\section{Ход работы}

	В условиях эксперимента необходимо учитывать фон, поэтому
	\begin{equation}
		N_0 = n_0 - n_\text{фон}, \ N_i = n_i - n_\text{фон}.
	\end{equation}
	
	\begin{table}[h!]
		\centering
		\begin{tabular}{| c | c | c | c |}
\hline
$$ & $n$ & $t, c$ & $N, \frac{part}{c}$\\
\hline
$N_0$ & $899735$ & $60$ & $14996$\\
\hline
$n_фон$ & $690$ & $60$ & $12$\\
\hline
\end{tabular}

		\caption{: измерения фона и числа частиц $N_0$ без поглотитела}
	\label{tb0}
	\end{table}

	\begin{table}[h!]
		\centering
		\begin{tabular}{| c | c | c | c | c | c | c |}
\hline
$$ & $$ & $$ & $$ & $$ & $$ & $$\\
\hline
$d, mm$ & $N_1$ & $N_2$ & $N_3$ & $t_1, c$ & $t_2, c$ & $t_3, c$\\
\hline
$20$ & $263952$ & $260973$ & $263118$ & $30$ & $30$ & $30$\\
\hline
$39,9$ & $160503$ & $159906$ & $160081$ & $30,51$ & $30,54$ & $30,72$\\
\hline
$59,9$ & $97798$ & $98489$ & $99099$ & $30,56$ & $30,56$ & $30,54$\\
\hline
$80$ & $59773$ & $60059$ & $62402$ & $30,43$ & $30,24$ & $30,63$\\
\hline
$100,1$ & $39806$ & $40409$ & $40142$ & $30,51$ & $30,5$ & $30,52$\\
\hline
$120$ & $25347$ & $25638$ & $25903$ & $30,45$ & $30,45$ & $30,45$\\
\hline
\end{tabular}

		\caption{: измерения числа частиц для алюминия}
	\label{tb0_5}
	\end{table}


	\begin{table}[h!]
		\centering
		\begin{tabular}{| c | c | c | c | c | c | c |}
\hline
$свинец$ & $$ & $$ & $$ & $$ & $$ & $$\\
\hline
$d, mm$ & $N_1$ & $N_2$ & $N_3$ & $t_1, c$ & $t_2, c$ & $t_3, c$\\
\hline
$4,7$ & $455151$ & $432762$ & $433034$ & $60$ & $60$ & $60$\\
\hline
$9,4$ & $213251$ & $211264$ & $213926$ & $60$ & $60$ & $60$\\
\hline
$14,4$ & $57808$ & $60097$ & $60509$ & $30$ & $30$ & $30$\\
\hline
$19$ & $31461$ & $31782$ & $32412$ & $30$ & $30$ & $30$\\
\hline
$24$ & $18439$ & $18414$ & $18550$ & $30,2$ & $30,57$ & $30,56$\\
\hline
\end{tabular}

		\caption{: измерения числа частиц для свинца}
	\label{tb1_5}
	\end{table}

	

	\begin{table}[h!]
		\centering
		\begin{tabular}{| c | c | c | c | c | c | c | c | c |}
\hline
$железо$ & $$ & $$ & $$ & $$ & $$ & $$ & $$ & $$\\
\hline
$d, mm$ & $N_1$ & $N_2$ & $N_3$ & $<N>$ & $N$ & $t_1, c$ & $t_2, c$ & $t_3, c$\\
\hline
$10$ & $115277$ & $113948$ & $114488$ & $114571$ & $7374$ & $15,42$ & $15,54$ & $15,65$\\
\hline
$20,1$ & $54906$ & $54438$ & $55587$ & $54977$ & $3549$ & $15,51$ & $15,55$ & $15,41$\\
\hline
$30,2$ & $27951$ & $28301$ & $28747$ & $28333$ & $1819$ & $15,45$ & $15,56$ & $15,71$\\
\hline
$40,3$ & $15106$ & $14870$ & $15330$ & $15102$ & $980$ & $15,44$ & $15,29$ & $15,49$\\
\hline
$50,4$ & $8412$ & $8315$ & $8274$ & $8334$ & $537$ & $15,54$ & $15,41$ & $15,6$\\
\hline
$60,4$ & $4564$ & $4560$ & $4553$ & $4559$ & $294$ & $15,52$ & $15,53$ & $15,45$\\
\hline
\end{tabular}

		\caption{: измерения числа частиц для железа}
	\label{tb2_5}
	\end{table}

	\newpage
	\section{Обработка результатов}

	\textbf{Погрешности}
	При повторном проведении опыта измренения количества частиц для фиксирванной длины замечаем, что измеренная величина отличается не более чем на 3\% от среднечего значения. 
	На самом деле среднее значение отклонения от среднего $\sim$ 0,5 $\div$ 1,5 \%. Но беру с запасом веррхнюю границу. Значит, 
	
	\[ \frac{\sigma_N}{N} \approx \frac{\sigma_{N_{0}}}{N_{0}} = 0,03 \]

	\[	\sigma_{\ln\left(\frac{N_0}{N}\right)} = \frac{1}{\frac{N_0}{N}} \sigma_{N_0/N} = \frac{1}{\frac{N_0}{N}} \cdot \frac{N_0}{N} \cdot \sqrt{\left(\frac{\sigma_{N_0}}{N_0}\right)^2+\left(\frac{\sigma_{N}}{N}\right)^2} \]

	\[  \sigma_{\ln\left(\frac{N_0}{N}\right)} \approx = 0,042 \]


	Погрешность измерения длины поглотителя - это просто погрешность штангециркуля, т.к. каждый измерения происходили после каждого добавления нового слоя.
	Таким образом, $\sigma_d = 0,01 см$
		
	
	Обрабтаем полученные во время работы данные:

	\begin{itemize}
		\item определим среднее число частиц 3-х измерений для фиксированной длины
		\item определим число частиц, детектируемое за 1 секунду
		\item учтем рассчитанную погрешность
		\item построим графики для свинца, железа, алюминия
	\end{itemize}

	\begin{table}[h!]
		\centering
		\begin{tabular}{| c | c | c | c | c |}
\hline
$U, mV$ & $T_{room}, K$ & $T, K$ & $T_{br}, K$ & $ \sigma_{T}, K$\\
\hline
$39920$ & $298$ & $973,66$ & $1000$ & $ 28 $\\
\hline
\end{tabular}

		\caption{: обработанные данные для алюминия}
	\label{tb1}
	\end{table}

	\begin{table}[h!]
		\centering
		\begin{tabular}{| c | c | c |}
\hline
$\lambda, A$ & $\theta$ & $\sigma_{\theta}$\\
\hline
$5790$ & $2456$ & $5$\\
\hline
$5461$ & $2282$ & $5$\\
\hline
$4395$ & $1866$ & $5$\\
\hline
$4047$ & $1222$ & $5$\\
\hline
\end{tabular}

		\caption{: обработанные данные для свинца}
	\label{tb2}
	\end{table}

	\begin{table}[h!]
		\centering
		\begin{tabular}{| c | c | c | c |}
\hline
$1/N(\theta)$ & $1 - cos(theta)$ & $\sigma_{1 - cos(\theta)}$ & $\sigma_{1/N(\theta}$\\
\hline
$0,0012$ & $0$ & $0$ & $0,000012$\\
\hline
$0,0011$ & $0,0152$ & $0,0061$ & $0,000011$\\
\hline
$0,0011$ & $0,0603$ & $0,0119$ & $0,000011$\\
\hline
$0,0014$ & $0,134$ & $0,0175$ & $0,000014$\\
\hline
$0,0015$ & $0,234$ & $0,0224$ & $0,000015$\\
\hline
$0,0018$ & $0,3572$ & $0,0267$ & $0,000018$\\
\hline
$0,002$ & $0,5$ & $0,0302$ & $0,00002$\\
\hline
$0,0023$ & $0,658$ & $0,0328$ & $0,000023$\\
\hline
$0,0025$ & $0,8264$ & $0,0344$ & $0,000025$\\
\hline
$0,0027$ & $1$ & $0,0349$ & $0,000027$\\
\hline
$0,003$ & $1,1736$ & $0,0344$ & $0,00003$\\
\hline
\end{tabular}

		\caption{: обработанные данные для железа}
	\label{tb3}
	\end{table}	
	
	

	\begin{table}[h!]
		\centering
		\begin{tabular}{| c | c | c | c |}
\hline
$d, cm$ & $ln(N_0/N)$ & $\sigma_{d}, см$ & $\sigma_{ln(N_0/N)}$\\
\hline
$2$ & $0,54$ & $0,01$ & $0,042$\\
\hline
$3,99$ & $1,05$ & $0,01$ & $0,042$\\
\hline
$5,99$ & $1,54$ & $0,01$ & $0,042$\\
\hline
$8$ & $2,02$ & $0,01$ & $0,042$\\
\hline
$10,01$ & $2,44$ & $0,01$ & $0,042$\\
\hline
$12$ & $2,89$ & $0,01$ & $0,042$\\
\hline
\end{tabular}

		\caption{: данные для графика, алюминий}
	\label{tb4}
	\end{table}

	\begin{table}[h!]
		\centering
		\begin{tabular}{| c | c | c | c |}
\hline
$d, cm$ & $ln(N_0/N)$ & $\sigma_{d}, см$ & $\sigma_{ln(N_0/N)}$\\
\hline
$0,47$ & $0,72$ & $0,01$ & $0,042$\\
\hline
$0,94$ & $1,44$ & $0,01$ & $0,042$\\
\hline
$1,44$ & $2,03$ & $0,01$ & $0,042$\\
\hline
$1,9$ & $2,66$ & $0,01$ & $0,042$\\
\hline
$2,4$ & $3,23$ & $0,01$ & $0,042$\\
\hline
\end{tabular}

		\caption{: данные для графика, свинец}
	\label{tb5}
	\end{table}

	\begin{table}[h!]
		\centering
		\begin{tabular}{| c | c | c | c |}
\hline
$d, cm$ & $ln(N_0/N)$ & $\sigma_{d}, см$ & $\sigma_{ln(N_0/N)}$\\
\hline
$1$ & $0,71$ & $0,01$ & $0,042$\\
\hline
$2,01$ & $1,44$ & $0,01$ & $0,042$\\
\hline
$3,02$ & $2,12$ & $0,01$ & $0,042$\\
\hline
$4,03$ & $2,74$ & $0,01$ & $0,042$\\
\hline
$5,04$ & $3,35$ & $0,01$ & $0,042$\\
\hline
$6,04$ & $3,97$ & $0,01$ & $0,042$\\
\hline
\end{tabular}

		\caption{: данные для графика, железо}
	\label{tb6}
	\end{table}

	Учтем погрешность МНК:

	\[ \sigma_{\mu} = \dfrac{1}{\sqrt{n}}\sqrt{\dfrac{\langle y^2 \rangle - \langle y \rangle^2}{\langle x^2 \rangle - \langle x \rangle^2} - \mu^{2}} \]

	Из построенных графиков по МНК получаем, что:

	\begin{table}[h!]
		\centering
		\begin{tabular}{| c | c | c | c | c | c |}
\hline
$\omega, c^{-1} \cdot 10^{15}$ & $a$ & $b$ & $V_0, В$ & $\sigma_{V_0}, В$ & $\sigma_{\omega}  c^{-1} \cdot 10^{15}$\\
\hline
$2,94$ & $0,04731$ & $0,01367$ & $0,28895$ & $0,004$ & $0,02$\\
\hline
$3,09$ & $0,04454$ & $0,01544$ & $0,34665$ & $0,004$ & $0,02$\\
\hline
$3,22$ & $0,03223$ & $0,01337$ & $0,41483$ & $0,004$ & $0,02$\\
\hline
$2,82$ & $0,02812$ & $0,00661$ & $0,23506$ & $0,004$ & $0,02$\\
\hline
$2,9$ & $0,03264$ & $0,00899$ & $0,27543$ & $0,004$ & $0,02$\\
\hline
\end{tabular}

		\caption{: результаты}
	\label{tb6}
	\end{table}

	Из таблицы 4 приложения V получаем соответствующую среднюю энерги $\gamma$ - лучей, испускаемых истчочником:

	\begin{itemize}
		\item $\mu_{Al} = 0,234 \pm 0,0038 \Rightarrow E_{\gamma} \approx 0,4 \div 0,5 \text{ МЭв} $
		\item $\mu_{Pb} = 1,23 \pm 0,053 \Rightarrow E_{\gamma} \approx 0,6 \text{ МЭв} $
		\item $\mu_{Fe} = 0,234 \pm 0,0038 \rightarrow E_{\gamma} \approx 0,4 \div 0,5 \text{ МЭв} $
	\end{itemize}

	\begin{figure}[h!]
		\centering
		\includegraphics[width=1.0\linewidth]{pics/al.png}
		\caption{ : зависимость для алюминия}
		\label{al}
	\end{figure}

	\begin{figure}[h!]
		\centering
		\includegraphics[width=1.0\linewidth]{pics/fr.png}
		\caption{ : зависимость для железа}
		\label{fr}
	\end{figure}

	\begin{figure}[h!]
		\centering
		\includegraphics[width=1.0\linewidth]{pics/pb.png}
		\caption{ : зависимость для свинца}
		\label{pb}
	\end{figure}

	\newpage

	\section{Вывод}
	В настоящей лабораторной работе с помощью сцинтилляционного счетчика были измерены линейные коэффициенты ослабления $\mu$ потока $\gamma$-лучей в свинце, железе, алюминии. 
	Среднюю энергию излучения, испускаемого источником, определили по справочной таблице, приведенной в приложении лабораторного практикума по общей физике.

	Приведу результаты еще раз дам небольшой анализ.

	\begin{itemize}
		\item $\mu_{Al} = 0,234 \pm 0,0038 \Rightarrow E_{\gamma} \approx 0,4 \div 0,5 \text{ МЭв} $
		\item $\mu_{Pb} = 1,23 \pm 0,053 \Rightarrow E_{\gamma} \approx 0,6 \text{ МЭв} $
		\item $\mu_{Fe} = 0,234 \pm 0,0038 \rightarrow E_{\gamma} \approx 0,4 \div 0,5 \text{ МЭв} $
	\end{itemize}

	\begin{enumerate}
		\item значения линейных коэффициентов, а также погрешности получились адекватного порядка.
		\item средняя энергия испускаемых квантов получилась \textbf{для каждого вещества} приблизительно порядка $\approx 0,5 \text{ МЭв}$
	\end{enumerate}

	Учитывая два пункта, приведенных выше, можно считать, что эксперимент проведён успешно и полученные значения в пределах погрешности соответствуют реальным значениям. 

\end{document}





