\documentclass[15pt,a5paper,reqno]{article}
\usepackage{hyperref}
\usepackage[warn]{mathtext}
\usepackage[utf8]{inputenc}
\usepackage[T2A]{fontenc}
\usepackage[russian]{babel}
\usepackage{amssymb, amsmath, multicol}
\usepackage{graphicx}
\usepackage[shortcuts,cyremdash]{extdash}
\usepackage{wrapfig}
\usepackage{floatflt}
\usepackage{lipsum}
\usepackage{verbatim}
\usepackage{concmath}
\usepackage{euler}
\usepackage{xcolor}
\usepackage{etoolbox}
\usepackage{fancyhdr}
\usepackage{subfiles}
\usepackage{enumitem}
\usepackage{amsthm}
\usepackage{indentfirst}
\usepackage{import}

\DeclareMathOperator{\sign}{sign}

\RequirePackage[ left     = 1.5cm,
  right    = 1.5cm,
  top      = 2.0cm,
  bottom   = 1.25cm,
  includefoot,
  footskip = 1.25cm ]{geometry}
\setlength    {\parskip}        { .5em plus .15em minus .08em }
%\setlength    {\parindent}      { .0em }
\renewcommand {\baselinestretch}{ 1.07 }

\fancyhf{}

\renewcommand{\footrulewidth}{ .0em }
\fancyfoot[C]{\texttt{\textemdash~\thepage~\textemdash}}
\fancyhead[R]{\hfilШурыгин}

\makeatletter
\patchcmd\l@section{%
  \nobreak\hfil\nobreak
}{%
  \nobreak
  \leaders\hbox{%
    $\m@th \mkern \@dotsep mu\hbox{.}\mkern \@dotsep mu$%
  }%
  \hfill
  \nobreak
}{}{\errmessage{\noexpand\l@section could not be patched}}
\makeatother
\parindent = 1cm % отступ при красной строке⏎
\pagestyle{fancy}    
\renewcommand\qedsymbol{$\blacksquare$}

\newcommand{\when}[2]{
  \left. #1 \right|_{#2} \hspace
}
\renewcommand{\kappa}{\varkappa}
\RequirePackage{caption2}
\renewcommand\captionlabeldelim{}
\newcommand*{\hm}[1]{#1\nobreak\discretionary{}

\DeclareSymbolFont{T2Aletters}{T2A}{cmr}{m}{it}
{\hbox{$\mathsurround=0pt #1$}}{}}
% Цвета для гиперссылок
\definecolor{linkcolor}{HTML}{000000} % цвет ссылок
\definecolor{urlcolor}{HTML}{799B03} % цвет гиперссылок
 
\hypersetup{pdfstartview=FitH,  linkcolor=linkcolor,urlcolor=urlcolor, colorlinks=true}


%\setcounter{secnum[utf8x]depth}{0}

\begin{document}

% НАЧАЛО ТИТУЛЬНОГО ЛИСТА
\begin{center}
  {\small ФЕДЕРАЛЬНОЕ ГОСУДАРСТВЕННОЕ АВТОНОМНОЕ ОБРАЗОВАТЕЛЬНОЕ\\ УЧРЕЖДЕНИЕ ВЫСШЕГО ОБРАЗОВАНИЯ\\ МОСКОВСКИЙ ФИЗИКО-ТЕХНИЧЕСКИЙ ИНСТИТУТ\\ (НАЦИОНАЛЬНЫЙ ИССЛЕДОВАТЕЛЬСКИЙ УНИВЕРСИТЕТ)\\ ФИЗТЕХ-ШКОЛА РАДИОТЕХНИКИ И КИБЕРНЕТИКИ}\\
  \hfill \break
  \hfill \break
  \hfill \break
  \Huge{Исследование эффекта Комптона.}\\
\end{center}

\hfill \break
\hfill \break
\hfill \break
\hfill \break
\hfill \break
\hfill \break

\begin{flushright}
  \normalsize{Работу выполнил:}\\
  \normalsize{\textbf{Шурыгин Антон Алексеевич, группа Б01-909}}\\
\end{flushright}

\begin{center}
  \normalsize{\textbf{Долгопрудный, 2021}}
\end{center}


\thispagestyle{empty} % выключаем отображение номера для этой страницы

% КОНЕЦ ТИТУЛЬНОГО ЛИСТА

\newpage
\thispagestyle{plain}
\tableofcontents
\thispagestyle{plain}
\newpage

\paragraph{Цель работы:} исследовать энергетический спектр $\gamma$-квантов, рассеянных на графите: определить количество рассеянных $\gamma$-квантов в зависимости от угла рассения; энергию покоя частиц, на которых происходит комптоновское рассeяние
\paragraph{Оборудование:} источник, графитовая мишень, сцинтилляционный спектрометр, фотоэлектронный умножитель, сцинтиллятор, коллиматор, лимб.


\section{Теория}

Пусть на покоящийся электрон (энергия покоя $mc^2$) налетает $\gamma$-квант с начальной энергией $\hbar \omega_0$. После соударения электрон приобретает энергию $\gamma mc^2$ и импульс $\gamma mv$, а $\gamma$-квант рассеивается на некотрый угол $\theta$ по отношению к начальному направлению с новой энергией $\hbar \omega_1$. Из законов сохранения импульса и энергии можно получить, что разница между длинами волн падающего и рассенного $\gamma$-квантов

\begin{equation}\label{1}
    \Delta \lambda = \lambda_1 - \lambda_0 = \dfrac{h}{mc} (1-\cos \theta).
\end{equation}

В наличии этой разницы и заключается эффект Комптона. Для дальнейшего применения полезно будет представить \eqref{1} в виде

\begin{equation}\label{2}
  \frac{1}{\varepsilon(\theta)} - \dfrac{1}{\varepsilon_0} = 1 - \cos \theta
\end{equation}


где $\varepsilon_0 = E_0/mc^2$ -- начальная энергия $\gamma$-квантов в единицах $mc^2$, $\varepsilon(\theta)$ -- энергия рассеянных $\gamma$ -квантов в тех же единицах.\\

Отметим, что всё вышесказанное применительно в том случае, когда электрон свободный, что справедливо для лёгких атомов, где энергия связи не больше нескольких килоэлектрон-вольт, а чаще всего меньше, и $\gamma$-квантов с энергией в несколько десятков-сотен килоэлектрон-вольт.

\section{Экспериментальная установка}



\begin{figure}[h!]
  \centering
  \includegraphics[width=1.0\linewidth]{pics/lab_5_1_2.png}
  \caption{Блок-схема установки}
\label{grph}
\end{figure}
Измерительный комплекс состоит из ФЭУ, питаемого от высоковольтного выпрямителя ВСВ, усилителя-анализатора УА, являющегося входным интерфейсом компьютера ЭВМ, управляемого с клавиатуры КЛ (Рис. 1). Информация отображается на дисплее Д. При работе ФЭУ в спектрометрическом режиме величина выходного электрического импульса пропорциональна энергии регистрируемого $\gamma$-кванта. В итоге возникает распределение электрических испульсов (Рис. 2), имеющее фотопик, положение вершины которого нас будет интересовать. Левее фотопика начинается непрерывный спектр комптоновских электронов, который сохраняется при любом угле рассеяния. Номер канала на распределении соответствует энергии регистрируемой частицы, точность его определения примерно $1\%$.

Пусть $\varepsilon(\theta) = AN(\theta)$, $A$ -- коэффициент пропорциональность, $N(\theta)$ -- номер соответствующего канала. Тогда \eqref{2} перепишется как

\begin{equation}\label{3}
  \dfrac{1}{N(\theta)} - \dfrac{1}{N(0)} = A(1-\cos \theta)
\end{equation}

Отсюда можно определить энергию покоя электрона как 

\begin{equation}\label{4}
    mc^2 = E_\gamma \dfrac{N(90)}{N(0) - N(90)},
\end{equation}

где $E_\gamma = E_0$ -- энергия испускаемых источником $\gamma$-квантов.

\section{Ход работы и обработка данных}

Устанавливая сцинтилляционный счётчик под разными углами $\theta$, произведём измерения, каждое примерно по десять минут, отмечая, какому каналу соответствует фотопик при каждом значении угла. Картина, наблюдаемая на дисплее компьютера, представлена на Рис. 3
Результаты измерений представлены в Таблице 1, как отмечалось выше, погрешнсть измерения канала -- 1\%, так как она для всех измерений больше, чем половина расстояния до соседнего возможного пика, учитывалась только она, погрешность измерения угла $\theta$ берём ценой деления лимба $\sigma_\theta = 2^\circ$.

\begin{table}[h!]
	\centering
	\begin{tabular}{| c | c | c |}
\hline
$\lambda, A$ & $\theta$ & $\sigma_{\theta}$\\
\hline
$5790$ & $2456$ & $5$\\
\hline
$5461$ & $2282$ & $5$\\
\hline
$4395$ & $1866$ & $5$\\
\hline
$4047$ & $1222$ & $5$\\
\hline
\end{tabular}

	\caption{: снятые показания}
\label{tb1}
\end{table}  

\begin{table}[h!]
	\centering
	\begin{tabular}{| c | c | c | c |}
\hline
$1/N(\theta)$ & $1 - cos(theta)$ & $\sigma_{1 - cos(\theta)}$ & $\sigma_{1/N(\theta}$\\
\hline
$0,0012$ & $0$ & $0$ & $0,000012$\\
\hline
$0,0011$ & $0,0152$ & $0,0061$ & $0,000011$\\
\hline
$0,0011$ & $0,0603$ & $0,0119$ & $0,000011$\\
\hline
$0,0014$ & $0,134$ & $0,0175$ & $0,000014$\\
\hline
$0,0015$ & $0,234$ & $0,0224$ & $0,000015$\\
\hline
$0,0018$ & $0,3572$ & $0,0267$ & $0,000018$\\
\hline
$0,002$ & $0,5$ & $0,0302$ & $0,00002$\\
\hline
$0,0023$ & $0,658$ & $0,0328$ & $0,000023$\\
\hline
$0,0025$ & $0,8264$ & $0,0344$ & $0,000025$\\
\hline
$0,0027$ & $1$ & $0,0349$ & $0,000027$\\
\hline
$0,003$ & $1,1736$ & $0,0344$ & $0,00003$\\
\hline
\end{tabular}

	\caption{: данные для графика}
\label{tb2}
\end{table}

По этим данным постоим график зависимости $1/N(\theta)$ от $1-\cos \theta$ (Рис. 4). Здесь погрешности считали по формулам

\begin{figure}[h!]
  \centering
  \includegraphics[width=1.0\linewidth]{pics/grph_5_1_2.png}
  \caption{Зависимость $\frac{1}{N(\theta)}$ от $1 - cos(\theta)$}
\label{grph}
\end{figure}

\[\begin{array}{l}
\sigma_{1/N} = \dfrac{\sigma_N}{N^2},\\
\sigma_{1-\cos \theta} = \sin(\theta) \sigma_\theta.\\
\end{array}\]


%Заметим, что во все формулы $\theta$ и $\sigma_\theta$ подставляется в радианах. Из графика по МНК получим угловой коэффициент, в соответствии с \eqref{1b} равный $A$, и точку пересечения с осью ординат, соответветствующую $1/N(0)$. Формулы расчёта (здесь $x \eqdef 1 -\cos\theta$ и $y \eqdef 1/N$):

\[\begin{array}{l l}
A = \dfrac{\langle xy \rangle - \langle x \rangle \langle y \rangle}{\langle x^2 \rangle - \langle x \rangle^2}, & \sigma_A = \dfrac{1}{\sqrt{n}}\sqrt{\dfrac{\langle y^2 \rangle - \langle y \rangle^2}{\langle x^2 \rangle - \langle x \rangle^2} - A^2},\\
\end{array}\]


Из аппроксимации получим <<наилучшие>> значения каналов для $\theta = 0^\circ$ и $\theta = 90^\circ$:

\[\begin{array}{l}
N_{\text{best}}(0) = \dfrac{1}{\frac{1}{N(0)}} = 877 \pm 9 \\
N_{\text{best}}(90) = \dfrac{1}{\frac{1}{N(0)}+A} = 333 \pm 3\\
\end{array}
\]

где погрешности считались по формулам
 
\[
\begin{array}{l}
\sigma_{N_{\text{best}}(0)} = \dfrac{\sigma_{\frac{1}{N(0)}}}{(\frac{1}{N(0)})^2},\\[14pt]
\sigma_{N_{\text{best}}(90)} = \dfrac{\sigma_{\frac{1}{N(0)}} + \sigma_{A}}{(\frac{1}{N(0)}+A)^2} \\
\end{array}
\]

Наконец, по формуле \eqref{4} получим энергию покоя электрона
    
    \[ mc^2 = 408 \pm 14~\text{кэВ} \]

погрешность считалась по формуле

\[\sigma_{mc^2} = \sqrt{ \left( \dfrac{\partial (mc^2)}{\partial N_{\text{best}}(0)} \right)^2 \sigma_{N_{\text{best}}(0)}^2 +\left( \dfrac{\partial (mc^2)}{\partial N_{\text{best}}(90)} \right)^2 \sigma_{N_{\text{best}}(90)}^2 }\]

Здесь использовалось, что $E_\gamma = 662~\text{кэВ}$. Истинная энергия покоя электрона 510 кэВ лежит в двух сигмах от полученного результата.

\section{Вывод}

В ходе работы было исследовано рассеяние $\gamma$-квантов на графите, подтверждена теоретическая формула для распределения энергии $\gamma$-квантов по углам рассеяния, а также как следствие посчитана энергия покоя электрона $mc^2 = 408 \pm 14 ~\text{кэВ}$.


\end{document}