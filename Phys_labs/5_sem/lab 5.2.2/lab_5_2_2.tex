\documentclass[a4paper, 14pt]{extarticle}%тип документа

\date{}

\usepackage{graphicx}
\usepackage{cmap}
\usepackage[T2A]{fontenc}
\usepackage[utf8]{inputenc}
\usepackage{indentfirst}


\usepackage[english,russian]{babel}
\usepackage{multirow} % Слияние строк в таблице
\newcommand
{\un}[1]
{\ensuremath{\text{#1}}}

%Русский язык
\usepackage[T2A]{fontenc} %кодировка
\usepackage[utf8]{inputenc} %кодировка исходного кода
\usepackage[english,russian]{babel} %локализация и переносы

%Таблицы
\usepackage[table,xcdraw]{xcolor}
\usepackage{booktabs}

%Математика
\usepackage{amsmath, amsfonts, amssymb, amsthm, mathtools}

%отступы 
\usepackage[left=2cm,right=2cm,top=2cm,bottom=3cm,bindingoffset=0cm]{geometry}

%Графики
\usepackage{pgfplots}
\pgfplotsset{compat=1.9}

%Вставка картинок
\usepackage{graphicx}
\usepackage{wrapfig, caption}
\graphicspath{}
\DeclareGraphicsExtensions{.pdf,.png,.jpg, .jpeg}
\newcommand\ECaption[1]{%
     \captionsetup{font=footnotesize}%
     \caption{#1}}

\begin{document}

	% НАЧАЛО ТИТУЛЬНОГО ЛИСТА

\begin{titlepage}
	\begin{center}
		\large 	Московский физико-технический университет \\
		Физтех-школа радиотехники и компьютерных технологий\\
		\vspace{0.2cm}
		
		\vspace{4.5cm}
		Лабораторная работа № 1.2.3 \\ \vspace{0.2cm}
		\LARGE \textbf{Измерение магнитного поля Земли}
	\end{center}
	\vspace{2.3cm} \large
	
	\begin{center}
		Работу выполнил: \\
		Шурыгин Антон \\
		Б01-909

	\end{center}
	
	\begin{center} \vspace{60mm}
		г. Долгопрудный \\
	\end{center}
\end{titlepage}


	\textbf{Цель работы:} исследовать сериальные закономерости в оптическом спектре водорода; спектр поглощения паров йода в видимой области.
	
    
\section{Введение и краткая теория}

Интерферометр Майкельсона находит применение в спектрометрах с высоким разрешением, для абсолютных и относительных измерений длин с точностью 0,005 мкм. 

Оптическая схема интерферометра приведена на рис. 1. Источником света служит лазер $ЛГ$. Лазер излучает узкий пучок света, который фокусируется линзой Л1. В фокусе этой линзы возникает точечный
источник света S. Сферическая световая волна от источника S падает на делительный кубик ДК и делится его диагональной гранью на 
две волны — отражённую $1$ и проходящую $2$. Волна 1 отражается от
зеркала $З_1$, возвращается к кубику, частично проходит сквозь него и
попадает на экран $Э$. Волна $2$ отражается от зеркала $З_2$, частично отражается от кубика и также попадает на экран. Световые волны $1$ и $2$
испускаются одним источником S, и они когерентны между собой. Эти
волны создают на экране $Э$ интерференционную картину. Для увеличения масштаба интерференционной картины может быть использована
линза $Л_2$.

Зеркало З1 установлено перпендикулярно падающему лучу. Оно может перемещаться вдоль луча. Это зеркало в дальнейшем будет называться подвижным. Зеркало З2 вдоль направления падающего луча не
перемещается. Его, однако, можно наклонять по отношению к лучу.

\begin{figure}[h!]
    \centering
    \includegraphics[width=1.2\linewidth]{pics/scheme.png}
    \caption{Схема интерферометра}
    \label{}
\end{figure}


       
        \section{Экспериментальная установка}
        Для измерения длин волн спектральных линий в работе используется стеклянный-призменный монохроматор-спектрометр УМ-2, предназначенный для спектральных исследований.
        

        \begin{figure}[h!]
            \centering
            \includegraphics[width=0.6\linewidth]{pics/scheme1.png}
            \caption{Устройство монохроматора УМ-2}
            \label{scheme1}
        \end{figure} 
        
        В работе спектр поглощения паров йода наблюдается визуально на фоне сплошного спектра лампы накаливания, питаемой от блока питания.
        
        
        \begin{figure}[h!]
            \centering
            \includegraphics[width=0.6\linewidth]{pics/scheme2.png}
            \caption{Схема экспериментальной установки для изучения спектра поглощения паров йода}
            \label{scheme2}
        \end{figure} 
        
        \newpage

        \section{Ход работы и обработка данных}
        
        Калибровка спектрометра была выполнена по спектрам ртути и неона. По ртути следует калиброваться в коротковолновой части спектра, а по неону -- в средней и длинноволновой.

        \begin{table}[h!]
            \centering
            \begin{tabular}{| c | c | c | c | c |}
\hline
$U, mV$ & $T_{room}, K$ & $T, K$ & $T_{br}, K$ & $ \sigma_{T}, K$\\
\hline
$39920$ & $298$ & $973,66$ & $1000$ & $ 28 $\\
\hline
\end{tabular}

            \caption{: калибровка для неона}
        \label{tb1}
        \end{table}
    
        \begin{table}[h!]
            \centering
            \begin{tabular}{| c | c | c |}
\hline
$\lambda, A$ & $\theta$ & $\sigma_{\theta}$\\
\hline
$5790$ & $2456$ & $5$\\
\hline
$5461$ & $2282$ & $5$\\
\hline
$4395$ & $1866$ & $5$\\
\hline
$4047$ & $1222$ & $5$\\
\hline
\end{tabular}

            \caption{: калибровка для ртути}
        \label{tb2}
        \end{table}

        Измерим положения трёх линий водорода из серии Бальмера --- $H_{\alpha}, H_{\beta}, H_{\gamma}$. Линии $H_{\delta}$ и более коротковолновые пронаблюдать не удалось ввиду их слабой интенсивности.	Получили соответствующие показания спектрометра:
		
		\[	H_{\alpha}: (2792\pm 5) \] 
        \[  H_{\beta} : (1814 \pm 5) \]
        \[ H_{\gamma} : (1172 \pm 5) \]

		Проградуируем спектрометр, для чего используем спектры неоновой и ртутной лампы, длины волн спектральных линий которых известны. 
		
        \begin{figure}[h!]
            \centering
            \includegraphics[width=1.0\linewidth]{pics/grad.png}
            \caption{ : градуировка спектрометра}
            \label{al}
        \end{figure}
	
        Получаем приближение полиноном второй степени: 
        
        \[ f(x) = 0,0007x^{2} -1,1725x + 4444 \]


		С учётом градуировки спектрометра получаем следующие длины волн для водорода: 
        
		
		\[	H_{\alpha} = (650\pm 15)\ \text{нм} \]
        \[  H_{\beta} = (486\pm 20)\ \text{нм} \] 
        \[  H_{\gamma} = (430\pm  25)\  \text{нм} \]
		
		
		Для каждой линии определим константу Ридберга по формуле (\ref{eq:Ry}), учитывая, что $m=2$, $Z=1$, а также, что:
        
        \newpage

        \begin{itemize}
            \item для линии $H_{\alpha} \Rightarrow n=3$
            \item для линии $H_{\beta}  \Rightarrow n=4$
            \item для линии $H_{\gamma} \Rightarrow n=5$
        \end{itemize}
        Получаем следующие значения константы Ридберга:
		
	    \[ \text{Ry}_{\alpha}=(1.115\pm 0.04) \cdot 10^{-2} \ \text{нм}^{-1} \] 
        \[ \text{Ry}_{\beta} =(1.10\pm 0.05)\cdot 10^{-2} \ \text{нм}^{-1} \] 
        \[ \text{Ry}_{\gamma}=(1.11\pm 0.06)\cdot 10^{-2}\ \text{нм}^{-1} \]
		
	
		Возьмем среднее среди полученных значений константы Ридберга и определим её экспериментально полученное значение:
		
		\[ \text{Ry}_E=(1.10\pm 0.05)\cdot 10^{-2} ~\text{нм}^{-1} \]

		Полученное значение вполне совпадает с табличным значением в пределах погрешности:

		\[ \text{Ry}=1.097\cdot 10^{-2} \ \text{нм}^{-1} \]
		
		Запишем показания спектрометра для следующих переходов в молекуле йода: 
        \begin{itemize}
            \item $\theta_{1,0}$ -- переход из первого колебательного уровня основного состояния в нулевой колебательный уровень возбуждённого состояния
            \item $\theta_{1,5}$ -- переход из первого колебательного уровня основного состояния в пятый колебательный уровень возбуждённого состояния
            \item $\theta_{g}$ -- переход из нулевого колебательного уровня основного состояния в область непрерывного спектра возбуждённого состояния
        \end{itemize}
        
        Получаем следующие данные:
		
		\[	\theta_{1,0}=(2700\pm 5), \] 
        \[  \theta_{1,5}=(2620\pm 5), \] 
        \[  \theta_g=(2000\pm 5)  \]
		
		 откуда находим соответствующие длины волн: 
		 
		 \begin{equation*}
		 	\lambda_{1,0}=(620\pm 30) \ \text{нм}, \ \lambda_{1,5}=(610\pm 30) \ \text{нм}, \ \lambda_g=(510\pm 30)\ \text{нм}.
		 \end{equation*}
		
		Определим энергию колебательного кванта возбуждённого состояния молекулы по формуле: 
		\begin{equation*}
			h \nu_2=\dfrac{h\nu_{1,5}-h \nu_{0,5}}{5}.
		\end{equation*}
		Итого:
		
		\[	h\nu_2=(1.0\pm 0.2)\cdot 10^{-2} \ \text{эВ} \]
			
		Вычислим:
        \begin{itemize}
            \item энергию электронного перехода $\Delta E=E_2-E_1$
            \item энергию диссоциации $D_1$ в основном состоянии
            \item энергию диссоциации $D_2$ в возбуждённом состоянии
        \end{itemize}

        При условии, что известны энергия колебательного кванта основного состояния есть $h\nu_1=0,027$~эВ, 
        энергия возбуждения, то есть энергия перехода атома из области непрерывного спектра основного состояния в область непрерывного спектра возбуждённого состояния, равна $E_A=0.94$ эВ.\\
		
        \newpage

        Имеем систему уравнений:

		\begin{equation*}
			\begin{cases}
				D_1+E_A=h \nu_g,\\

				h\nu_g=D_2+\Delta E,\\

				h\nu_{1,0}=\Delta E+h\nu_2-\dfrac{3}{2}h\nu_1,\\

				h\nu_{1,5}=\Delta E+\dfrac{11}{2}h\nu_2-\dfrac{3}{2}h\nu_1.

			\end{cases}
		\end{equation*}

		Из неё находим все необходимые величины:
		\[ \Delta E=(2.0\pm 0.1) \ \text{эВ} \] 
        \[  D_1=(1.5\pm 0.1)\  \text{эВ}  \] 
        \[ D_2=(0.42\pm 0.1) \ \text{эВ} \]

    \section{Вывод}
	В работе исследовались сериальные закономерности в оптическом спектре водорода и спектр поглощения паров йода в видимой области.
	
    Была построена градуировочная кривая по данным спектрам неона и ртути. Затем получены длины волн линий $H_{\alpha}$, $H_{\beta}$ и $H_{\gamma}$ серии Бальмера:
    
    \[	H_{\alpha} = (650\pm 15)\ \text{нм} \]
    \[  H_{\beta} = (486\pm 20)\ \text{нм} \] 
    \[  H_{\gamma} = (430\pm  25)\  \text{нм} \]
		
		
    Вычислена постоянная Ридберга:
    
    \[ \text{Ry}_p=(1.10\pm 0.05)\cdot 10^{-2} ~\text{нм}^{-1} \]

    В пределах погрешности экспериментальное значение в пределах погрешности совпадает с теоретическим:

    \[ \text{Ry}_t=1.097\cdot 10^{-2} \ \text{нм}^{-1} \]
    
	
	Получены длины волн, соответствующие некоторым электронно-колебательным переходам из основного состояния в возбуждённое. Вычислены энергия колебательного кванта возбуждённого состояния молекулы, 
    
    \[	h\nu_2=(1.0\pm 0.2)\cdot 10^{-2} \ \text{эВ} \]
    
    Энергия электронного перехода:

    \[ \Delta E=(2.0\pm 0.1) \ \text{эВ} \] 
    
    Энергии диссоциации молекулы в основном и в возбуждённом состояниях:
	
    \[  D_1=(1.5\pm 0.1)\  \text{эВ}  \] 
    \[ D_2=(0.42\pm 0.1) \ \text{эВ} \]


\end{document}