\documentclass[a4paper, 14pt]{extarticle}%тип документа

\date{}

\usepackage{graphicx}
\usepackage{cmap}
\usepackage[T2A]{fontenc}
\usepackage[utf8]{inputenc}
\usepackage{indentfirst}


\usepackage[english,russian]{babel}
\usepackage{multirow} % Слияние строк в таблице
\newcommand
{\un}[1]
{\ensuremath{\text{#1}}}

%Русский язык
\usepackage[T2A]{fontenc} %кодировка
\usepackage[utf8]{inputenc} %кодировка исходного кода
\usepackage[english,russian]{babel} %локализация и переносы

%Таблицы
\usepackage[table,xcdraw]{xcolor}
\usepackage{booktabs}

%Математика
\usepackage{amsmath, amsfonts, amssymb, amsthm, mathtools}

%отступы 
\usepackage[left=2cm,right=2cm,top=2cm,bottom=3cm,bindingoffset=0cm]{geometry}

%Графики
\usepackage{pgfplots}
\pgfplotsset{compat=1.9}

%Вставка картинок
\usepackage{graphicx}
\usepackage{wrapfig, caption}
\graphicspath{}
\DeclareGraphicsExtensions{.pdf,.png,.jpg, .jpeg}
\newcommand\ECaption[1]{%
     \captionsetup{font=footnotesize}%
     \caption{#1}}

\begin{document}

% НАЧАЛО ТИТУЛЬНОГО ЛИСТА

\begin{titlepage}
	\begin{center}
		\large 	Московский физико-технический университет \\
		Физтех-школа радиотехники и компьютерных технологий\\
		\vspace{0.2cm}
		
		\vspace{4.5cm}
		Лабораторная работа № 1.2.3 \\ \vspace{0.2cm}
		\LARGE \textbf{Измерение магнитного поля Земли}
	\end{center}
	\vspace{2.3cm} \large
	
	\begin{center}
		Работу выполнил: \\
		Шурыгин Антон \\
		Б01-909

	\end{center}
	
	\begin{center} \vspace{60mm}
		г. Долгопрудный \\
	\end{center}
\end{titlepage}


\textbf{Цель работы:}
Определить фокусные расстояния собирающих и рассеивающих линз, смоделировать ход лучей в трубе Галилея, трубе Кеплера и микроскопе, определить их увеличение

\textbf{Оборудование}: оптическая скамья, набор линз, экран, осветитель со шкалой, 
зрительная труба, диафрагма, линейка.


\section{Введение и краткая теория}

Интерферометр Майкельсона находит применение в спектрометрах с высоким разрешением, для абсолютных и относительных измерений длин с точностью 0,005 мкм. 

Оптическая схема интерферометра приведена на рис. 1. Источником света служит лазер $ЛГ$. Лазер излучает узкий пучок света, который фокусируется линзой Л1. В фокусе этой линзы возникает точечный
источник света S. Сферическая световая волна от источника S падает на делительный кубик ДК и делится его диагональной гранью на 
две волны — отражённую $1$ и проходящую $2$. Волна 1 отражается от
зеркала $З_1$, возвращается к кубику, частично проходит сквозь него и
попадает на экран $Э$. Волна $2$ отражается от зеркала $З_2$, частично отражается от кубика и также попадает на экран. Световые волны $1$ и $2$
испускаются одним источником S, и они когерентны между собой. Эти
волны создают на экране $Э$ интерференционную картину. Для увеличения масштаба интерференционной картины может быть использована
линза $Л_2$.

Зеркало З1 установлено перпендикулярно падающему лучу. Оно может перемещаться вдоль луча. Это зеркало в дальнейшем будет называться подвижным. Зеркало З2 вдоль направления падающего луча не
перемещается. Его, однако, можно наклонять по отношению к лучу.

\begin{figure}[h!]
    \centering
    \includegraphics[width=1.2\linewidth]{pics/scheme.png}
    \caption{Схема интерферометра}
    \label{}
\end{figure}



\section{Определение фокусных расстояний линз с помощью зрительной трубы}
\subsection{Определение фокусного расстояния собирающих линз}


Настроим зрительную трубу на бесконечность
Поставим положительную линзу на расстоянии от предмета примерно равном фокусному. На небольшом расстоянии от линзы закрепим трубу, настроенную на бесконечность,
и отцентрируем её по высоте. Диафрагма диаметром $d = 1$ см, надетая на ближнюю к осветителю линзу, уменьшит сферические аберрации и повысит чёткость изображения.

Передвигая линзу вдоль скамьи, получим в окуляре зрительной трубы изображение предмета — миллиметровой сетки. При этом расстояние между предметом и серединой тонкой линзы (между проточками на оправах) равно фокусному.

    \begin{table}[h!]
        \centering
            \begin{tabular}{| c | c | c | c |}

                \hline
                    $n$ & $F_1, cm$ & $F_2, cm$ &  тип линзы\\
                \hline
                    1 & 8 & 8.2 & соб. \\
                \hline
                    2 & 10 & 10 & соб.\\
                \hline
                    3 & 18,8 & 19,3  & соб. \\
                \hline
                    4 & 32,5 & 32,5 & соб. \\
                \hline
                    5 & -9 & -9 &  рас.\\
                \hline
                \end{tabular}
        \caption{Результаты измерения фокусных расстояний линз}
        \label{nu1}
    \end{table}
    

\subsection{Определение фокусного расстояния рассеивающей линзф}

Для определения фокусного расстояния тонкой отрицательной линзы сначала получим на экране увеличенное изображение сетки при помощи одной короткофокусной положительной линзы. Измерим расстояние между линзой и экраном $a_0 = 33.7$ см.
Разместим сразу за экраном трубу, настроенную на бесконечность, и закрпим её. Уберём экран и поставьте на его место исследуемую рассеивающую линзу (рис. 8). Перемещая рассеивающую линзу, найдите в окуляре зрительной трубы резкое изображение сетки. \par
Измерив расстояние между линзами $l = 24.7$ см, рассчитаем фокусное расстояние рассеивающей линзы $f = a_0 - l$.
Результаты измерения фокусного расстояния рассеивающих линз:

\begin{center}
    $f_{рас} = 33,7 - 24,7 \approx 9$ см
\end{center}


За погрешность измерений берем предел в 0.5 см, что логично из таблицы (1). Кроме того наложим ограничение в $0.05F$ для получения четкого изображения.

\begin{figure}[h!]
    \begin{center}
        \begin{minipage}[h!]{0.60\linewidth}
            \includegraphics[width=1\linewidth]{pics/plus_lens.png}
            \caption{Определение фокусного расстояния собирающей линзы} %% подпись к рисунку
        \label{} 
        \end{minipage}

        \hfill 

        \begin{minipage}[h!]{0.60\linewidth}
            \includegraphics[width=1\linewidth]{pics/minus_lens.png}
            \caption{Определение фокусного расстояния рассеивающей линзы}
        \label{}
    \end{minipage}
    \end{center}
\end{figure}

\section{Моделирование трубы Кеплера}

Рассмотрим ход лучей в трубе Кеплера и найдём увеличение данной оптической системы:
    
\begin{figure}[h!]
    \centering
        \includegraphics[width=10cm]{pics/kepler.png}
    \caption{Ход лучей в трубе Кеплера}
    \label{}
\end{figure}

Пусть пучок света, попадающий в объектив, составляет с оптической осью угол $\varphi_1$, а пучок, выходящий из окуляра, — угол $\varphi_2$. Увеличение $\gamma$ зрительной трубы по определению равно

\begin{equation}
    N = \frac{\tan \varphi_2}{\tan \varphi_1},
\end{equation}

но также из рис. 3 следует, что :

\begin{equation}
    N_{\tau} = -\frac{f_1}{f_2} = -\frac{D_1}{D_2},
\end{equation}

где $D_1$ - ширина пучка, прошедшего через объектив, а $D_2$ - ширина пучка, вышедшего из окуляра

Построим оптическую систему из каллиматора и непосредственно трубы Кеплера. 

\begin{figure}[h!]
        \centering
            \includegraphics[width=10cm]{pics/kepler_2.png}
            \caption{Схема трубы Кеплера}
        \label{}
\end{figure}

Параметры действующих линз:

\begin{center}
    $f_{1} = 32.5$ см \hspace{1cm} $f_2 = 8$ см
\end{center}

Найдём увеличение трубы Кеплера непосредственно: пусть $h_1$ - размер ячейки миллиметровой сетки без телескопа, $h_2$ - с телескопом

\begin{center}
	$h_1 = 9$ мал. дел., \hspace{1cm} $h_2 = 34$ мал. дел. \par

	$N_{\tau} = -\frac{h_2}{h_1} \approx 3,7 \pm 0,53$
\end{center}

При этом по формуле (2) также

\begin{center}
    $N_{\tau} = -\frac{f_{1}}{f_2} \approx 4,06 \pm 0,51$
\end{center}

Кроме того существует еще один способ найти увеличение: частное $D_1 $ - диаметр объектива и $D_2$ - диаметр его изображения.

\begin{center}
	$N_{\tau} = -\frac{D_2}{D_1} = -\frac{3,4}{0,8} \approx -4,25 \pm 0,62$
\end{center}



\section{Моделирование трубы Галилея}
    \begin{figure}[h!]
    \centering
    \includegraphics[width=9cm]{pics/gal.png}
    \caption{Ход лучей в трубе Галилея}
    \label{}
\end{figure}

Труба Галилея получается из трубы Кеплера заменой собирающей линзы окуляра рассеивающей. Формулы для увеличения, соответственно, остаются теми же:
\begin{equation}
    N_{\tau} = -\frac{f_1}{f_2} = -\frac{D_1}{D_2} =  -\frac{h_2}{h_1}
\end{equation}


Заменим собирающую линзу с фокусным расстоянием $f_2 = 8$ см рассеивающей с фокусным расстоянием $f_2 = 9$ см. Проведём те же операции, что и для трубы Кеплера:

\begin{center}
$h_1 = 9$ малых дел., \hspace{1cm} $h_2 = 30$ малых дел. \par
$N_{\tau} = -\frac{h_2}{h_1} \approx -3,33 \pm 0,46$
\end{center}

При этом выполняется так же:

\begin{center}
    $N_{\tau} = -\frac{f_1}{f_2} \approx -3,61 \pm 0,43$
\end{center}

Полученные значения хорошо совпадают.


\section{Моделирование микроскопа}

Для создания модели микроскопа с увеличением $N_M = -5$ (минус, т.к. изображение перевернуто) отберём самые короткофокусные линзы из набора. 
Рассчитаем необходимый интервал $\Delta$ и длину тубуса  $l_12$ по формулам:

\[ N_m = -\frac{\Delta}{f_1} \cdot \frac{L}{f_2} \: \rightarrow \: \Delta = -\frac{N_m f_1 f_2}{L}  \]


Где $L = 25$ см - расстояние наилучшего зрения нормального. 

\begin{figure}[h!]
    \centering
    \includegraphics[width=9cm]{pics/micro.png}
    \caption{Ход лучей в микроскопе}
    \label{}
\end{figure}


Получаем, что $\Delta = 16$ см. Затем $l_12 = \Delta + f_1 + f_2$ = 34 см.
Таким образом:

\[   N_m = -\frac{h_2 L}{h_1 f} = -\frac{31 \cdot 25}{9 \cdot 18,8} \approx -4,58  \pm 1,02  \]


   
\section{Вывод}
В ходе работы были определены фокусные расстояния собирающих и рассеивающих линз с помощью зрительной трубы. 
\newline
Из этих линз далее сконструированы следующие оптические приборы: труба Кеплера, труба Галилея, микроскоп. 
\newline
Были определены их увеличения и проведено сравнение с её действительным значением. В пределах погрешности теоретическое значение хорошо совпадает с практическим.

\end{document}

