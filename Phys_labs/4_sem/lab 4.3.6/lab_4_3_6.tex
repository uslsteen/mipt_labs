\documentclass[a4paper, 14pt]{extarticle}%тип документа

\date{}

\usepackage{graphicx}
\usepackage{cmap}
\usepackage[T2A]{fontenc}
\usepackage[utf8]{inputenc}
\usepackage{indentfirst}


\usepackage[english,russian]{babel}
\usepackage{multirow} % Слияние строк в таблице
\newcommand
{\un}[1]
{\ensuremath{\text{#1}}}

%Русский язык
\usepackage[T2A]{fontenc} %кодировка
\usepackage[utf8]{inputenc} %кодировка исходного кода
\usepackage[english,russian]{babel} %локализация и переносы

%Таблицы
\usepackage[table,xcdraw]{xcolor}
\usepackage{booktabs}

%Математика
\usepackage{amsmath, amsfonts, amssymb, amsthm, mathtools}

%отступы 
\usepackage[left=2cm,right=2cm,top=2cm,bottom=3cm,bindingoffset=0cm]{geometry}

%Графики
\usepackage{pgfplots}
\pgfplotsset{compat=1.9}

%Вставка картинок
\usepackage{graphicx}
\usepackage{wrapfig, caption}
\graphicspath{}
\DeclareGraphicsExtensions{.pdf,.png,.jpg, .jpeg}
\newcommand\ECaption[1]{%
     \captionsetup{font=footnotesize}%
     \caption{#1}}

\begin{document}
  
% НАЧАЛО ТИТУЛЬНОГО ЛИСТА

\begin{titlepage}
	\begin{center}
		\large 	Московский физико-технический университет \\
		Физтех-школа радиотехники и компьютерных технологий\\
		\vspace{0.2cm}
		
		\vspace{4.5cm}
		Лабораторная работа № 1.2.3 \\ \vspace{0.2cm}
		\LARGE \textbf{Измерение магнитного поля Земли}
	\end{center}
	\vspace{2.3cm} \large
	
	\begin{center}
		Работу выполнил: \\
		Шурыгин Антон \\
		Б01-909

	\end{center}
	
	\begin{center} \vspace{60mm}
		г. Долгопрудный \\
	\end{center}
\end{titlepage}

\textbf{Цель работы:}
Изучение явления саморепродукции и применение его к измерению параметров периодических структур.

\textbf{Оборудование:} лазер,кассета с сетками, мира, короткофокусная линза с микрометрическим винтом, экран, линейка.


\section{Введение и краткая теория}

Интерферометр Майкельсона находит применение в спектрометрах с высоким разрешением, для абсолютных и относительных измерений длин с точностью 0,005 мкм. 

Оптическая схема интерферометра приведена на рис. 1. Источником света служит лазер $ЛГ$. Лазер излучает узкий пучок света, который фокусируется линзой Л1. В фокусе этой линзы возникает точечный
источник света S. Сферическая световая волна от источника S падает на делительный кубик ДК и делится его диагональной гранью на 
две волны — отражённую $1$ и проходящую $2$. Волна 1 отражается от
зеркала $З_1$, возвращается к кубику, частично проходит сквозь него и
попадает на экран $Э$. Волна $2$ отражается от зеркала $З_2$, частично отражается от кубика и также попадает на экран. Световые волны $1$ и $2$
испускаются одним источником S, и они когерентны между собой. Эти
волны создают на экране $Э$ интерференционную картину. Для увеличения масштаба интерференционной картины может быть использована
линза $Л_2$.

Зеркало З1 установлено перпендикулярно падающему лучу. Оно может перемещаться вдоль луча. Это зеркало в дальнейшем будет называться подвижным. Зеркало З2 вдоль направления падающего луча не
перемещается. Его, однако, можно наклонять по отношению к лучу.

\begin{figure}[h!]
    \centering
    \includegraphics[width=1.2\linewidth]{pics/scheme.png}
    \caption{Схема интерферометра}
    \label{}
\end{figure}



\section{Схема установки}

\begin{figure}[h!]
    \centering
    \includegraphics[width=15cm]{pics/scheme.png}
    \caption{Схема лабораторной установки}
    \label{fig:vac}
\end{figure}

\section{Ход работы}


\subsection{Измерение периода решеток по их пространственному спектру}


\begin{table}[h!]
	\centering
	\begin{tabular}{| c | c | c | c | c | c |}
    \hline
    $n_{сетки}$ & $X_m$, мм & $m$ & $x$, мм  & $d$, мм & $\sigma_d$, мм\\
    \hline
    1 & 201  & 6 & 33.50  &  0.020 & <0.001\\
    \hline
    2 & 223 & 9 & 24.77  &   0.027 & <0.001\\
    \hline
    3 & 177  & 16 & 11.1 &   0.061 & 0.001\\
    \hline
    4 & 235  & 24 & 9.79   &  0.069 & 0.001\\
    \hline
    5 & 63  & 16 & 3.93    &   0.174  & 0.007\\
    \hline
    \end{tabular}
    
	\caption{Измерение расстояние между соседними дифр. макс. на экране}
	\label{nu1}
\end{table}


Расстояние от кассеты до экрана $L = 124 \: cm$, $\lambda = 560 \: nm$.

\begin{equation}
    dsin(\theta_x) = m_x \lambda, \:\:\:\: d sin(\theta_y) = m_y \lambda     
  \label{first}
\end{equation}

Полагая $sin(\theta) \backsimeq \theta \backsimeq \frac{x}{L}$, найдем с помощью формул (1) период каждой решетки.

\[     d = \frac{\lambda L}{x}        \]

Измерения и результаты вычисления периоды дифракционной решетки занесены в таблицу 1.



\subsection{Измерение периода решеток по изображению, увеличенного с помощью линзы}

Найдем период решетки другим способом.
 
\begin{table}[h!]
	\centering
	\begin{tabular}{| c | c | c | c | c |}
\hline
$n_сетки$ & $x_m, mm$ & $m$ & $D, mm$ & $d, mm$\\
\hline
1 & - & - & - & -\\
\hline
2 & 3,5 & 6 & 0,58 & 0,027 \\
\hline
3 & 9 & 6 & 1,5 & 0,072\\
\hline
4 & 12 & 4 & 3 & 0,144\\
\hline
5 & 16 & 4 & 4 &  0,192\\
\hline
\end{tabular}

	\caption{Определение размера клеток $D$}
	\label{tb3}
\end{table}


Измеренные расстояния: между сеткой и экраном - $a' = 131 \: cm \rightarrow$ между линзой и сеткой -  $a = a' - b = 6 \: cm$ , между линзой и экраном - $ b= 125\:cm$.

\[    d = \frac{a}{b} \cdot D   \]

Таким образом по формуле выше находим период решетки и записываем результат в таблицу 2. 


\subsection{Исследование саморепродукции с помощью сеток}

Исследуем саморепродукцию.
Находим координаты $z_n$ плоскостей саморепродукции, строим график. По коэффициенту наклона прямой графика определим период решетки по формуле:

\begin{equation}
  d_i = \sqrt{\frac{k_i\lambda}{2}}
\end{equation}


\begin{table}[h!]
	\centering
	\begin{tabular}{| c | c | c | c | c | c |}
    \hline
        $n$ & $z_1, mm$ & $z_2, mm$ & $z_3, mm$ & $z_4, mm$  & $z_5, mm$\\
    \hline
        1 & - & 4 & 8,1 & 6,4 & 21,5\\
    \hline
        2 & - & 8,55 & 15,1 & 22,1 & 43,5\\
    \hline
        3 & - & 12,2 & 22 & 33,2 & 65,5\\
    \hline
        4 & - & 17,55 & 28,5 & 49,2 & -\\
    \hline
        5 & - & 20,55 & 35,2 & 59,7 & -\\
    \hline
        6 & - & 23,6 & 42 & - & -\\
    \hline
        7 & - & 28,8 & 49 & - & -\\
    \hline
        8 & - & - & 55,5 & - & -\\
  \hline
  \end{tabular}

	\caption{Измерение номера дифракционной картины от координаты линзы}
	\label{tb1}
\end{table}

\begin{table}[h!]
	\centering
	\begin{tabular}{| c | c | c | c | c | c |}
    \hline
    $n_сетки$ & 1 & 2 & 3 & 4 & 5  \\
    \hline
    $d, mm$   & - & 0,034 & 0,043 & 0,061 & 0,077   \\
    \hline
\end{tabular}
	\caption{Резульаты вычисления периода дифракционных решеток}
	\label{tb2}
\end{table}

Измерения и полученные значения сводим в таблицу 3. 
Затем строим графики $z = f(n)$.


\begin{figure}[h!]
  \centering
  \includegraphics[width=13cm]{pics/lab_436_1.png}
  \caption{График $z_1 = f(n)$}
  \label{}
\end{figure}

\begin{figure}[h!]
  \centering
  \includegraphics[width=13cm]{pics/lab_436_2.png}
  \caption{График $z_2 = f(n)$}
  \label{}
\end{figure}


\begin{figure}[h!]
  \centering
  \includegraphics[width=13cm]{pics/lab_436_3.png}
  \caption{График $z_3 = f(n)$}
  \label{}
\end{figure}


\begin{figure}[h!]
  \centering
  \includegraphics[width=13cm]{pics/lab_436_4.png}
  \caption{График $z_4 = f(n)$}
  \label{}
\end{figure}


\subsection{Исследование миры}

Измеряем расстояние между экраном и линзой - $L_3$, экраном и мирой - $L_4$.
Получаем, что $L_3 = 126 cm$, $L_4 = 132 cm$.

Ширина штриха миры равна $1 mm$.




\begin{table}[h!]
	\centering
	\begin{tabular}{| c | c | c |}
\hline
$n$ & $z_n (25), mm$ & $z_n (20), mm$\\
\hline
-3 & 17 & 12,2\\
\hline
-2 & 20 & 17,2\\
\hline
-1 & 23 & 22,5\\
\hline
0 & 26,1 & 28\\
\hline
1 & 28,3 & 33\\
\hline
2 & 31,9 & 38\\
\hline
3 & 34,5 & 43,5\\
\hline
4 & 37,5 & 48,5\\
\hline
5 & 41,1 & -\\
\hline
\end{tabular}

	\caption{Исследование решеток миры}
	\label{tb2_1}
\end{table}

\end{document}