\documentclass[a4paper, 14pt]{extarticle}%тип документа
\date{}

\usepackage{indentfirst}


\usepackage{graphicx}
\usepackage{cmap}
\usepackage[T2A]{fontenc}
\usepackage[utf8]{inputenc}

%%\renewcommand{\footrulewidth}{ .0em }
\usepackage[english,russian]{babel}
\usepackage{multirow} % Слияние строк в таблице
\newcommand
{\un}[1]
{\ensuremath{\text{#1}}}


%Русский язык
\usepackage[T2A]{fontenc} %кодировка
\usepackage[utf8]{inputenc} %кодировка исходного кода
\usepackage[english,russian]{babel} %локализация и переносы

%Таблицы
\usepackage[table,xcdraw]{xcolor}
\usepackage{booktabs}

%Математика
\usepackage{amsmath, amsfonts, amssymb, amsthm, mathtools}

%отступы 
\usepackage[left=2cm,right=2cm,top=2cm,bottom=3cm,bindingoffset=0cm]{geometry}

%Вставка картинок
\usepackage{graphicx}
\usepackage{wrapfig, caption}
\graphicspath{}
\DeclareGraphicsExtensions{.pdf,.png,.jpg, .jpeg}
\newcommand\ECaption[1]{%
     \captionsetup{font=footnotesize}%
     \caption{#1}}

%Таблицы
\usepackage[table,xcdraw]{xcolor}
\usepackage{booktabs}

%Графики
\usepackage{pgfplots}
\pgfplotsset{compat=1.9}
\usepackage[english,russian]{babel}
\usepackage{multirow} % Слияние строк в таблице
\newcommand
{\un}[1]
{\ensuremath{\text{#1}}}

\begin{titlepage}
	\begin{center}
		\large 	Московский физико-технический университет \\
		Физтех-школа радиотехники и компьютерных технологий\\
		\vspace{0.2cm}
		
		\vspace{4.5cm}
		Лабораторная работа № 16 \\ \vspace{0.2cm}
		\LARGE \textbf{Шумы в электронных схемах}
	\end{center}
	\vspace{2.3cm} \large
	
	\begin{center}
		Работу выполнили: \\
		Тяжкороб Ульяна  \\
		Шурыгин Антон \\
		Широкова Ксения \\
		Б01-909
	\end{center}
	
	
	\begin{center} \vspace{40mm}
		г. Долгопрудный \\
		2020 \\
	\end{center}
\end{titlepage}


\begin{document}


\section{Тепловой шум}

\subsection{задание}

\textbf{Пункт 1}
\newline
Модель резистора как источник шумового напряжения.
Из графиков видно, чnто напряжения на входе/выходе равны $e = 4.0n$, поскольку коэффициент передачи равен 1. 
Варьируем: $R[1k, 16k | \log2$]
\newline
1) $R = 1k \Rightarrow e_1 = 4n$
\newline
2) $R = 2k \Rightarrow e_2 = 5.75n$
\newline
3) $R = 4k \Rightarrow e_4 = 8.15$
\newline
Тогда получим:
$\frac{e_2}{e_1}$ $\approx 1,43 \approx \sqrt{\frac{2k}{1k}} \approx \sqrt{2} \approx 1.41$

$\frac{e_4}{e_2}$ $\approx 1,42 \approx \sqrt{\frac{4k}{2k}} \approx \sqrt{2} \approx 1.41$

Значит, шум растет как \sqrt{R}.

\textbf{Пункт 2}
\newline
Подключим график из корня из интеграла от спектральной плотности, поставив номер в поле З, и измерив эффективное напряжение (уровень) шума $\sigma$ на выводах резисторов $R[1k, 16k | \log2]$, $R[1k, 1000k | \log10]$ в полосе $F = 1 MHz$.
\newline

\[ \sigma = \sqrt{\int_0^F n_e^{2}(f)df} \]

\begin{figure}[h!]
			\centering
			\includegraphics[width=1.1\linewidth]{pic1.jpg}
			\caption{Задание 1.1, пункт 2, Варьирование 1}
			\label{A}
\end{figure}

Первое варьирование: $R[1k, 16k | \log2]$
\newline
1) $R = 16k \Rightarrow \sigma = 16.138u$
\newline
2) $R = 8k \Rightarrow \sigma = 11.638u$
\newline
3) $R = 4k \Rightarrow \sigma = 8.224u$
\newline
4) $R = 2k \Rightarrow \sigma = 5.741u$
\newline
5) $R = 1k \Rightarrow \sigma = 4.198u$
\newline


\begin{figure}[h!]
	\centering
			\includegraphics[width=1.1\linewidth]{pic2.jpg}
			\caption{Задание 1.1, пункт 2, Варьирование 2}
	\label{A}
\end{figure}

Второе варьирование: $R[1k, 1000k | \log10]$
\newline
1) $R = 1k \Rightarrow \sigma = 4.138u$
\newline
2) $R = 10k \Rightarrow \sigma = 12.931u$
\newline
3) $R = 100k \Rightarrow \sigma = 40.086u$
\newline
4) $R = 1000k \Rightarrow \sigma = 128.017u$
\newline

\textbf{Пункт 3}
\newline
Затем переходим к модели источника точка: \{Is/ni\}

\begin{figure}[h!]
			\centering
			\includegraphics[width=1.1\linewidth]{pic3.jpg}
			\caption{Задание 1.1, пункт 3, Варьирование 1 для модели \{Is/ni\}}
			\label{A}
\end{figure}


Первое варьирование: $R[1k, 16k | \log2$]
\newline
1) $R = 1k \Rightarrow \sigma = 4.034u$
\newline
2) $R = 2k \Rightarrow \sigma = 5.741u$
\newline
3) $R = 4k \Rightarrow \sigma = 8.224u$
\newline
4) $R = 8k \Rightarrow \sigma = 11.328u$
\newline
5) $R = 16k \Rightarrow \sigma = 16.293u$
\newline

Замечаем, что напряжение растет как $\sqrt{R}$:
\newline
1) При $R_1 = 1k \Rightarrow  V_1 = 4.0n$
\newline
2) При $R_2 = 2k \Rightarrow  V_2 = 6.0n$
\newline
3) При $R_3 = 4k \Rightarrow  V_3 = 8.0n$
\newline
4) При $R_4 = 8k \Rightarrow  V_4 = 11.311n$
\newline
5) При $R_5 = 16k \Rightarrow  V_5 = 16.0n$
\newline
1) При $R_1 = 1k \Rightarrow  I_1 = 4.072p$
\newline
2) При $R_2 = 2k \Rightarrow  I_2 = 2.835p$
\newline
3) При $R_3 = 4k \Rightarrow  I_3 = 1.960p$
\newline
4) При $R_4 = 8k \Rightarrow  I_3 = 1,392p$
\newline
5) При $R_5 = 16k \Rightarrow  I_4 = 1.031p$
\newline

Замечаем, что напряжение растет как $\sqrt{R}$.
Например, $\frac{V_5}{V_4}$ $\approx 1,41 \approx \sqrt{\frac{16k}{8k}} \approx \sqrt{2} \approx 1.41$. Аналогично проверяем для остальных значений.


Заметим, что ток падает как $\frac{1}{\sqrt{R}}$.
Например, $\frac{I_2}{I_1} \approx 0.7$, при этом $\frac{R_1}{R_2} \approx \frac{1}{\sqrt{2}} \approx 0,72$. Аналогично проверяем для остальных значений.


\subsection{Задание}

\textbf{Пункт 1}
Р-трим модель с последовательным соединением $R_1, \: R_2$.
\newline
Проверим закон сложения шумовых напряжений.
\[ e(f) = \sqrt{4kT(R_1 + R_2)} \]
\newline
1) \textbf{Первое варьирование}: $R_1[0k, 1k | 1k]$, причем $R_2 = 2k$.
\newline
При $R_1 = 0k, \Rightarrow  e = 5.8n$.
Теоретическое значение $e \approx 5.75n$
\newline
При $R_1 = 1k, \Rightarrow  e = 7.00$
Теоретическое значение $e \approx 7.05n$
\newline
2) \textbf{Второе варьирование}: $R_2[0, 2k | 2k]$, причем $R_1 = 1k$.
\newline
При $R_2 = 0k, \Rightarrow  e = 4.00n$
Теоретическое значение $e \approx 4.07n$
\newline
При $R_2 = 2k, \Rightarrow  e = 7,05n$
Теоретическое значение $e \approx 7,00n$
\newline
Теоретические значения хорошо совпали с экспериментальными.
\newline
\begin{figure}[h!]
			\centering
			\includegraphics[width=1.1\linewidth]{1.2/pic4.jpg}
			\caption{Задание 1.2, пункт 1, Варьирование 1 для модели \{Es/se\}}
			\label{A}
\end{figure}
\newline
\begin{figure}[h!]
			\centering
			\includegraphics[width=1.1\linewidth]{1.2/pic5.jpg}
			\caption{Задание 1.2, пункт 1, Варьирование 2 для модели \{Es/se\}}
			\label{A}
\end{figure}

\newline

\textbf{Пункт 2}
Р-трим модель с параллельным соединением $R_3, \: R_4$ \{Is/ni\}
\newline
Проверим закон сложения шумовых токов:
\[ i(f) = \sqrt{\frac{4kT}{R_1 || R_2}}  \]
\newline
1) \textbf{Первое варьирование}: $R_3[1k, 100k | 99k]$, причем $R_4 = 2k$.
\newline
При $R_3 = 1k, \Rightarrow  i = 4,98p$.
Теоретическое значение $e \approx i = 5p$.
\newline
При $R_3 = 1k, \Rightarrow  i = 2.91p$.
Теоретическое значение $e \approx i = 2.94p$.
\newline
2) \textbf{Второе варьирование}: $R_4[2k, 100k | 98k]$, причем $R_3 = 1k$.
\newline
При $R_4 = 2k, \Rightarrow  i = 4.98p$.
Теоретическое значение $e \approx i = 5.00p$.
\newline
При $R_4 = 100k, \Rightarrow  i = 4.07p$.
Теоретическое значение $e \approx i = 4.09p$.
\newline
Теоретические значения хорошо совпали с экспериментальными.

\newline
\begin{figure}[h!]
			\centering
			\includegraphics[width=1.1\linewidth]{1.2/pic6.jpg}
			\caption{Задание 1.2, пункт 2, Варьирование 1 для модели \{Is/ni\}}
			\label{A}
\end{figure}
\newline
\begin{figure}[h!]
			\centering
			\includegraphics[width=1.1\linewidth]{1.2/pic7.jpg}
			\caption{Задание 1.2, пункт 2,Варьирование 2 для модели \{Is/ni\}}
			\label{A}
\end{figure}
\newline

\subsection{Задание}
\textbf{Пункт 1}
\begin{figure}[h!]
			\centering
			\includegraphics[width=1.1\linewidth]{1.3/pic8.jpg}
			\caption{Задание 1.3, пункт 1, данные для таблицы}
			\label{A}
\end{figure}

\begin{figure}[h!]
			\centering
			\includegraphics[width=1.1\linewidth]{1.3/pic9.jpg}
			\caption{Задание 1.3, пункт 1, График 1, $T_n(R)$}
			\label{A}
\end{figure}

\begin{figure}[h!]
			\centering
			\includegraphics[width=1.1\linewidth]{1.3/pic10.jpg}
			\caption{Задание 1.3, пункт 1, График 2, $K_n(R)$}
			\label{A}
\end{figure}



\begin{table}[h!]
  \caption{Таблица для построения}
  \begin{center}
  	\begin{tabular}{|c|c|c|c|}
  	    \hline
  	$R_s, \: kOm$ & $e_n, \: nV$ & $K_n$ & $T_n, \: K$ \\
  	    \hline
  	8 & 25.680 &  & 1193.34   \\
  		\hline
  	8 & 19.794 &  & 587.23    \\
  		\hline
  	8 & 16.082  &   & 285.66   \\
  		\hline
    8 & 13.918  &   &  138.65   \\
        \hline
    8 & 12.526  &   & 55.30    \\
        \hline
  	\end{tabular}
  \end{center}
\label{B_table}
\end{table}


\newpage
\newpage
\section{Дробовой шум}
\textbf{Пункт 1}
\begin{figure}[h!]
			\centering
			\includegraphics[width=1.1\linewidth]{2.1/pic12.jpg}
			\caption{Задание 2.1, пункт 1}
			\label{A}
\end{figure}

\newline
1)$I_{01} = 1m \Rightarrow e_1 = 17.8p$
\newline
2)$I_{02} = 100u \Rightarrow e_2 = 5.7p$
\newline
3)$I_{03} = 10u \Rightarrow e_3 = 1.8p$
\newline
4)$I_{04} = 1u \Rightarrow e_4 = 0.5p$
\newline

Проверим, что $\frac{e_1}{e_2} = \frac{\sqrt{I_{02}}}{\sqrt{I_{01}}}$
\newline

\begin{figure}[h!]
			\centering
			\includegraphics[width=1.1\linewidth]{2.1/pic11.jpg}
			\caption{Задание 2.1, пункт 2}
			\label{A}
\end{figure}

\newline
1) $I_{01} = 32m \Rightarrow e_1 = 98.5p$
\newline
2) $I_{02} = 16m \Rightarrow e_2 = 67.5p$
\newline
3) $I_{03} = 8m \Rightarrow  e_3 = 48.8p$
\newline
4) $I_{04} = 4m \Rightarrow e_4 = 35.0p$
\newline
5) $I_{05} = 2m \Rightarrow e_4 = 24.8p$
\newline
6) $I_{06} = 1m \Rightarrow e_4 = 17.5p$
\newline

Проверим, что $\frac{e_1}{e_2} = \frac{\sqrt{I_{02}}}{\sqrt{I_{01}}}$

\textbf{Пункт 2}
Учтем, что $r_D \approx K\cdot R_1$, причем $R_1 = 10K$
\newline
1)$I_{01} = 1u, K = 838.2m \Rightarrow r_{D_{01}} = 8.38K$
\newline
2)$I_{02} = 10u, K = 338.8m \Rightarrow r_{D_{02}} = 3.38K$
\newline
3)$I_{03} = 100u, K = 46.7m  \Rightarrow r_{D_{03}} = 460$
\newline
4)$I_{04} = 1m, K = 2.9m \Rightarrow r_{D_{04}} = 29$
\newline

\begin{figure}[h!]
			\centering
			\includegraphics[width=1.1\linewidth]{2.1/pic13.jpg}
			\caption{Задание 2.1, пункт 2}
			\label{A}
\end{figure}


\textbf{Пункт 3}
\newline
1) $I_1 = 1u, \Rightarrow e = 14.6n$
\newline
2) $I_1 = 10u, \Rightarrow e = 4.6n $
\newline
3) $I_1 = 100u , \Rightarrow  e = 1.4n $
\newline
4) $I_1 = 1m , \Rightarrow e = 420.6p$
\newline
5) $I_1 = 10m,  \Rightarrow e = 93.5p$
\newline
Проверяем формулу $e(f) = i(f)\cdot r_d$, при варьированиее $I_1[1u, 10m | Log10] $.
\newline
Причем $i(f) = \sqrt{2qI_0}$.
\newline
1) $i_1(f) = 5.65\cdot 10^{-13} \Rightarrow e = R_{D_{01}} \cdot i_1(f)  = 4.7n$
\newline
1) $i_1(f) = 1.78\cdot 10^{-12} \Rightarrow e = R_{D_{02}} \cdot i_1(f) = 6n$





\newpage

\section{Фильтрация шумов}
\subsection{задание}
\textbf{Пункт 1}
\newline

Открываем файл model3. 
Устанавливаем интегрирующую цепь.
По графику рис.14 определяем граничную частоту $f_h$.
Получаем $f_h = 10148 Hz$.


\begin{figure}[h!]
			\centering
			\includegraphics[width=1.1\linewidth]{3/3_1_1.jpg}
			\caption{Задание 3.1, пункт 1}
			\label{A}
\end{figure}

\textbf{Пункт 2}
\newline

Переключимся на шумовые графики. Измерим шумовое напряжение $n_1$ в полосе пропускания, уровень шума на выходе $\sigma$. 
\newline
Проверим, что $\sigma = n_1\cdot \sqrt{F_n} = \sqrt{\frac{kT}{C}}$, причем $F_n = \frac{\pi}{2} \cdot f_n$.

Измерения, вычисления так же приведены на скриншоте.
\newline
$n_1 = 12.874n$
\newline
$\sigma_{pract} = 1.604$, $\sigma_{theor} = 1.627$.


\begin{figure}[h!]
			\centering
			\includegraphics[width=1.1\linewidth]{3/3_1_2.jpg}
			\caption{Задание 3.1, пункт 2}
			\label{A}
\end{figure}


\textbf{Пункт 3}
\newline

Варьирование $R_1[2k, 16k | 4k]$.
Снимаем зависимость шумового напряжения от $R_1$.

\begin{figure}[h!]
			\centering
			\includegraphics[width=1.1\linewidth]{3/3_1_4.jpg}
			\caption{Задание 3.1, пункт 3}
			\label{A}
\end{figure}


\textbf{Пункт 4}
\newline

Варьирование $C_1[0.8n, 2.4n | 0.4k]$.
Cнимаем зависимость уровня шума от емкости.

\begin{figure}[h!]
			\centering
			\includegraphics[width=1.1\linewidth]{3/3_1_3.jpg}
			\caption{Задание 3.1, пункт 4}
			\label{A}
\end{figure}

\subsection{задание}

Теперь рассматрим полосовой LC-фильтр.

\textbf{Пункт 1}
\newline

По графику оцениваем резонансную частоту и полосу пропускания по уровню 0,7. Оцениваем добротность. Получаем:
\newline
$f_0 = 100694 \: Hz$, $\Delta f = 20626 \: Hz$, $Q = \frac{f_0}{\Delta f} \approx 4.8$.

\begin{figure}[h!]
			\centering
			\includegraphics[width=1.1\linewidth]{3/3_2_1.jpg}
			\caption{Задание 3.2, пункт 1}
			\label{A}
\end{figure}

\textbf{Пункт 2}
\newline

Переключимся на шумовые графики. Измеряем шумовое напряжение $n_2$ в точке $f_0$, уровень шума на выходе.
Проверяем аналогичные условия как в 3.1 пункт 1.

Измерения:
\newline
$n_2 = 10.219n$
\newline
$\sigma = 1.82u$.
\newline
Проверка формулы дала такое же значение шума.


\begin{figure}[h!]
			\centering
			\includegraphics[width=1.1\linewidth]{3/3_2_2.jpg}
			\caption{Задание 3.2, пункт 2}
			\label{A}
\end{figure}


\textbf{Пункт 3}
\newline

Варьирование $R_2[2.3k, 10.3k | 4k]$.
Снимаем зависимость шумового напряжения от $R_2$. Соответственные кривые подписаны на графике. 

\begin{figure}[h!]
			\centering
			\includegraphics[width=1.1\linewidth]{3/3_2_5.jpg}
			\caption{Задание 3.2, пункт 3}
			\label{A}
\end{figure}

\textbf{Пункт 4}
\newline

Варьирование $С_2[0.75n, 1.75n | 0.5n]$.

Замечаем, что при мзенении емкости меняется частота резонанса, но шумовое напряжение $n_2$ на этой частоте остается прежним.
Снимаем зависимость уровня шума на выходе $\sigma$ от емкости. 
Значение емкости указано для каждой кривой на скриншоте.

\begin{figure}[h!]
			\centering
			\includegraphics[width=1.1\linewidth]{3/3_2_3.jpg}
			\caption{Задание 3.2, пункт 4}
			\label{A}
\end{figure}


\textbf{Пункт 5}
\newline

Варьирование $L_2[1m, 3m | 1m]$.
Убеждаемся, что при изменении индуктивности сохраняются шумовое напряжение на частоте резонанса, уровень шума $\sigma$ на выходе. Для каждой кривой подписано соответственное значение индуктивности.

\begin{figure}[h!]
			\centering
			\includegraphics[width=1.1\linewidth]{3/3_2_4.jpg}
			\caption{Задание 3.2, пункт 5}
			\label{A}
\end{figure}

\subsection{задание}

Рассматриваем LC - фильтр нижниъ частот. 
\newline
$p = \frac{jf}{f_o}, f_0 = 100k, \rho = 1260, Q = \frac{1}{2\sigma} = 5$
\newline


\textbf{Пункт 1}
\newline

Измеряем шумовое напряжение $n_3$ в максимуме $f_0$, на частоте $f_0/10$, а так же уровень шума на выходе $\sigma$.
Проверим формулу $F_n = \frac{\pi\cdot f_0}{2\cdot Q}$.
\newline
Показания:
\newline
1) $f_0 = 99.54K \Rightarrow n_3 = 10.35n$
\newline
2) $f_0/10 = 9.95 \Rightarrow n_3 = 2.055n$
\newline
Напряжение шума на выходе:
\newline
$\sigma = 1.82u$.
\newline
Оценка шумовой полосы $F_n$ и проверка форумлы:
\newline
3) $F_{n_{theor}} = 31.3k$
\newline
4) $F_{n_{pract}} = (\frac{\sigma}{n_3})^{2} = 30.9k$

\begin{figure}[h!]
			\centering
			\includegraphics[width=1.1\linewidth]{3/3_3_1.jpg}
			\caption{Задание 3.3, пункт 1}
			\label{A}
\end{figure}



\begin{figure}[h!]
			\centering
			\includegraphics[width=1.1\linewidth]{3/3_3_2.jpg}
			\caption{Задание 3.3, пункт 1}
			\label{A}
\end{figure}



\textbf{Пункт 2}

Варьирование $R_3[100, 400 | 150]$.
\newline
Варьирование $C_3[0.75n, 1.75n | 0.5n]$.
\newline
Варьирование $L_3[1m, 3m | 1m]$.
\newline
Фиксируем зависимости $n_3(f_0), \: n_3(f_0/10), \: \sigma$ от изменяемых параметров.


\begin{figure}[h!]
			\centering
			\includegraphics[width=1.1\linewidth]{3/3_3_5.jpg}
			\caption{Задание 3.3, пункт 2, варьирование R}
			\label{A}
\end{figure}

\newline

\begin{figure}[h!]
			\centering
			\includegraphics[width=1.1\linewidth]{3/3_3_3.jpg}
			\caption{Задание 3.3, пункт 2, варьирование C}
			\label{A}
\end{figure}

\newline

\begin{figure}[h!]
			\centering
			\includegraphics[width=1.1\linewidth]{3/3_3_4.jpg}
			\caption{Задание 3.3, пункт 2, варьирование L}
			\label{A}
\end{figure}

\subsection{задание}

Рассмотрим LC - фильтр верхних частов с параметрами секции 3.3.
\newline

\textbf{Пункт 1}
\newline

Измерим шумовое напряжение $n_4$ в максимуме при $f_0$ и на частоте $10f_0$, уровень шума $\sigma$ на выходе в полосе $1MHz$.
\newline
1) $f_0 = 99.54K \Rightarrow n_4 = 10.35n $
\newline
2) $10f_0 = 990.54k \Rightarrow n_4 = 2.056n$
\newline
3) $\sigma = 2.692u$
\newline

\begin{figure}[h!]
			\centering
			\includegraphics[width=1.1\linewidth]{3/3_3_6.jpg}
			\caption{Задание 3.4, пункт 1}
			\label{A}
\end{figure}

\textbf{Пункт 2}

Варьирование $R_4[100, 400 | 150]$.
\newline
Варьирование $C_4[0.75n, 1.75n | 0.5n]$.
\newline
Варьирование $L_4[1m, 3m | 1m]$.
\newline
Фиксируем зависимости $n_4(f_0), \: n_4(10f_0), \: \sigma$ от изменяемых параметров.

\begin{figure}[h!]
			\centering
			\includegraphics[width=1.1\linewidth]{3/3_4_3.jpg}
			\caption{Задание 3.4, пункт 2, варьирование R}
			\label{A}
\end{figure}


\begin{figure}[h!]
			\centering
			\includegraphics[width=1.1\linewidth]{3/3_4_2.jpg}
			\caption{Задание 3.4, пункт 2, варьирование С}
			\label{A}
\end{figure}


\begin{figure}[h!]
			\centering
			\includegraphics[width=1.1\linewidth]{3/3_4_2.jpg}
			\caption{Задание 3.4, пункт 2, варьирование L}
			\label{A}
\end{figure}




\newpage

\section{Шумящие фильтры}

\subsection{задание}
\textbf{Пункт 1}
\newline
Открыаем файл model4.
Установим $\{n1/V1\}$ : фильтр 1 с параметрами:
\newline

\[ f_0 = 100kHz, \rho = 1260, Q = 3\]
\newline

Подключив $v(n_1)/(v(V1)$ снимем АЧХ фильтра. Измерим резонансную частоту $f_0$, полосу по уровню 0.7, коэффициент передачи на резонансной и нулоевой частотах. 
\newline
Сравним с теорией.
\newline
1) $f_0 = 100k, \Delta f = 35k$.
\newline
2) По графику:$K{f_0} \approx 0.5$.
\newline
3) Теоретический расчет по формуле:
\newline
\[ K = \frac{0.5}{1 + jQ\cdot(\frac{f}{f_0} - \frac{f_0}{f})}\]
\newline
Получаем, что $K_1 = 0.5$.
\newline
С теорией совпало.

\begin{figure}[h!]
			\centering
			\includegraphics[width=1.1\linewidth]{4/4_1_2.jpg}
			\caption{Задание 4.1, пункт 1}
			\label{A}
\end{figure}


\textbf{Пункт 2}
\newline

Измерим уровни шумвого напряжения на частотах $f_0, f_0/10$. Так же заменим поочередно первый и второй резисторы нешумящим сопротивлением $H_1$, оценим вклад шумов $R_{s1}, R_1$ в шумовое напряжение и в уровень шума на выходе:\newline
1) $n_{f_0} = 1.32n, \: n_{\frac{f_0}{10}} = 1.865n, \: \sigma = 1.819u$
\newline
2) $n_{f_0} = 994p, \: n_{\frac{f_0}{10}} = 1.865p, \: \sigma = 5.88u$
\newline
3) $n_{f_0} = 994.8p, \: n_{\frac{f_0}{10}} = 35p, \: \sigma = 235n$

\begin{figure}[h!]
			\centering
			\includegraphics[width=1.1\linewidth]{4/4_1_3.jpg}
			\caption{Задание 4.1, пункт 2}
			\label{A}
\end{figure}




\begin{figure}[h!]
			\centering
			\includegraphics[width=1.1\linewidth]{4/4_1_5.jpg}
			\caption{Задание 4.1, пункт 2}
			\label{A}
\end{figure}



\begin{figure}[h!]
			\centering
			\includegraphics[width=1.1\linewidth]{4/4_1_4.jpg}
			\caption{Задание 4.1, пункт 2}
			\label{A}
\end{figure}


\textbf{Пункт 3}
\newline


По графику рис.36 оцениваем значение коэффициента шума на частотах $f_0$, \:, \: $f_0/10$ по формуле
\newline

\[  K_n = 20lg(\frac{e_n(f)}{\sqrt{4kTR}})   \]

\newline
1) $e_{f_0}= 112n \Rightarrow K_n(f0) = 3$
\newline
2) $e_{f_0/10}= 2.643n \Rightarrow K_n(f_0/10) = 35.6 $
\newline


\begin{figure}[h!]
			\centering
			\includegraphics[width=1.1\linewidth]{4/4_1_1.jpg}
			\caption{Задание 4.1, пункт 3}
			\label{A}
\end{figure}

\subsection{задание}
\textbf{Пункт 1}
\newline

Установим $\{n1/V1\}$ : фильтр 2 с $f_0 = 50kHz$.
\newline

Подключив $v(n_2)/(v(V_2)$ снимем АЧХ фильтра. Измерим резонансную частоту $f_0$, полосу по уровню 0.7, коэффициент передачи на резонансной и нулоевой частотах. 
\newline
Сравним с теорией.
\newline
1) $f_0 = 50k, \Delta f = 153k$.
\newline
2) По графику: $K_{f_0} \approx 1/3$.
\newline
3) Теоретический расчет по формуле:
\newline
\[ K = \frac{0.5}{1 + jQ\cdot(\frac{f}{f_0} - \frac{f_0}{f})}\]
\newline
Получаем, что $K = 0.34$.
\newline
С теорией совпало.


\begin{figure}[h!]
			\centering
			\includegraphics[width=1.1\linewidth]{4/4_2_2.jpg}
			\caption{Задание 4.2, пункт 1}
			\label{A}
\end{figure}


\textbf{Пункт 2}
\newline
Измерим уровни шумвого напряжения на частотах $f_0, \: 10f_0$. Так же заменим поочередно первый и второй резисторы нешумящим сопротивлением $H_1$, оценим вклад шумов $R_{s2}, R_2$ в шумовое напряжение и в уровень шума на выходе:\newline
1) $n_{f_0} = 5.929n, \: n_{10f_0} = 1.572n, \: \sigma = 2.858u$
\newline
2) $n_{f_0} = 4.976n, \: n_{10f_0} = 993pp, \: \sigma = 2.141u$
\newline
3) $n_{f_0} = 241p, \: n_{10f_0} = 251p, \: \sigma = 370n$


\begin{figure}[h!]
			\centering
			\includegraphics[width=1.1\linewidth]{4/4_2_3.jpg}
			\caption{Задание 4.2, пункт 2}
			\label{A}
\end{figure}



\begin{figure}[h!]
			\centering
			\includegraphics[width=1.1\linewidth]{4/4_2_5.jpg}
			\caption{Задание 4.2, пункт 2}
			\label{A}
\end{figure}




\begin{figure}[h!]
			\centering
			\includegraphics[width=1.1\linewidth]{4/4_2_4.jpg}
			\caption{Задание 4.2, пункт 2}
			\label{A}
\end{figure}

\textbf{Пункт 3}
\newline


По графику рис.41 оцениваем значение коэффициента шума на частотах $f_0$, $10f_0$, $f_0/100$ по формуле
\newline

\[  K_n = 20lg\frac{e_n(f)}{\sqrt{4kTR}}   \]


1) $e_{f_0}= 17.787n \Rightarrow K_n(f_0) = 7.8 $
\newline
2) $e_{f_0/100}= 726n \Rightarrow K_n(f_0/100) = 40 $
\newline
3) $e_{f_0/10}= 74.46n \Rightarrow K_n(f_0/10) = 20.2 $
\newline

\begin{figure}[h!]
			\centering
			\includegraphics[width=1.1\linewidth]{4/4_2_1.jpg}
			\caption{Задание 4.2, пункт 3}
			\label{A}
\end{figure}



\subsection{задание}

\textbf{Пункт 1}
\newline

Подключив $v(n_3)/(v(V3)$ снимем АЧХ фильтра. Измерим резонансную частоту $f_0$, полосу по уровню 0.7, коэффициент передачи на резонансной и нулоевой частотах. 
\newline
Сравним с теорией.
\newline
1) $f_0 = 102k, \: \Delta f = 34k$.
\newline
2) По графику:$K_{f_0} \approx 0.5$.
3) Теоретический расчет по формуле:
\newline
\[ K = \frac{0.5}{1 + jQ\cdot(\frac{f}{f_0} - \frac{f_0}{f})}\]
\newline
Получаем, что $K_1 = 0.5$.
\newline
С теорией совпало.

\begin{figure}[h!]
			\centering
			\includegraphics[width=1.1\linewidth]{4/4_3_2.jpg}
			\caption{Задание 4.3, пункт 1}
			\label{A}
\end{figure}


\textbf{Пункт 2}
\newline

Измерим уровни шумвого напряжения на частотах $f_0, f_0/100$. Так же заменим поочередно первый и второй резисторы нешумящим сопротивлением $H_1$, оценим вклад шумов $R_{s3}, R_3$ в шумовое напряжение и в уровень шума на выходе:\newline
1) $n_{f_0} = 7.963n, \: n_{\frac{f_0}{10}} = 1.841n, \: \sigma = 1.819u$
\newline
2) $n_{f_0} = 5.335n, \: n_{\frac{f_0}{10}} = 344p, \: \sigma = 1.265u$
\newline
3) $n_{f_0} = 344.8p, \: n_{\frac{f_0}{10}} = 944p, \: \sigma = 230n$
 
\begin{figure}[h!]
			\centering
			\includegraphics[width=1.1\linewidth]{4/4_3_3.jpg}
			\caption{Задание 4.3, пункт 2}
			\label{A}
\end{figure}



\begin{figure}[h!]
			\centering
			\includegraphics[width=1.1\linewidth]{4/4_3_4.jpg}
			\caption{Задание 4.3, пункт 2}
			\label{A}
\end{figure}

\begin{figure}[h!]
			\centering
			\includegraphics[width=1.1\linewidth]{4/4_3_5.jpg}
			\caption{Задание 4.3, пункт 2}
			\label{A}
\end{figure}


\textbf{Пункт 3}
\newline

По графику рис.46 оцениваем значение коэффициента шума на частотах $f_0$, $10f_0$, $f_0/100$ по формуле
\newline

\[  K_n = 20lg(\frac{e_n(f)}{\sqrt{4kTR}})   \]

\newline
1) $e_{f_0}= 68.329n \Rightarrow  K_n(f_0) = 18.6 $
\newline
2) $e_{f_0/100}= 15.806n \Rightarrow K_n(f_0/100) = 31.3$
\newline
3) $e_{10f_0}= 11.28n \Rightarrow  K_n(10f_0) = 15.6$
\newline

\begin{figure}[h!]
			\centering
			\includegraphics[width=1.1\linewidth]{4/4_3_1.jpg}
			\caption{Задание 4.3, пункт 3}
			\label{A}
\end{figure}


\newpage

\section{Шумы в усилителе на биполярном транзисторе}

\subsection{Задание 5.1}
\textbf{Пункт 1}
\newline

Установим ${E1/i}$
\newline
$I_c = 300u, \:B = 200Ohm, \: I1 = 13.5uA.$ 
\newline
По графику оценим значение $h21$ = 

\begin{figure}[h!]
			\centering
			\includegraphics[width=1.1\linewidth]{5/5_1_1.jpg}
			\caption{Задание 5.1, пункт 1}
			\label{A}
\end{figure}

\textbf{Пункт 2}
\newline

Снимем зависимость $i(f)$ от $H_s$, варьируя $H_s[10, 1000k|log10]$, $RB = 0$

\begin{figure}[h!]
			\centering
			\includegraphics[width=1.1\linewidth]{5/5_1_2.jpg}
			\caption{Задание 5.1, пункт 2}
			\label{A}
\end{figure}

$H_s = 1M, i = 2.3u$
\newline
$H_s = 100k, i = 224.3n$
\newline
$H_s = 10k, i = 27.7n$
\newline
$H_s = 1k, i = 8.8n$
\newline
$H_s = 100, i = 7.4n$
\newline
$H_s = 10, i = 7.3n$
\newline
Снимем зависимость $i(f)$ от $H_s$, варьируя $RB[0, 100|25]$, $H_s = 0$

\begin{figure}[h!]
			\centering
			\includegraphics[width=1.1\linewidth]{5/5_1_3.jpg}
			\caption{Задание 5.1, пункт 2}
			\label{A}
\end{figure}

\textbf{Пункт 3}
\newline

Повторим измерения для $I_c = 0.1mA$

\begin{figure}[h!]
			\centering
			\includegraphics[width=1.1\linewidth]{5/5_1_4.jpg}
			\caption{Задание 5.1, пункт 3}
			\label{A}
\end{figure}

  	
\subsection{Задание 5.2}
\textbf{Пункт 1}
\newline

Установим $r_b = 200Ohm$ транзистора Q2, установим $I_c = 300u$, выбрав $R_b = $, $R_c = $ ($I_c*R_c = 5V$)

\section{Шумы в усилителе на полевом транзисторе}

\subsection{Задание 6.1}
\textbf{Пункт 1}
\newline

Установим ${V/i}$, $U_p = 0.8$. Начальное значение $I_d = $
\newline
Варьируя $U_p[0.2,2|0.2]$, исследуем зависимость крутизны транзистора S от $U_p$

\begin{figure}[h!]
			\centering
			\includegraphics[width=1.1\linewidth]{6/6_1_1.jpg}
			\caption{Задание 6.1, пункт 1}
			\label{A}
\end{figure}

$U_p = 0.2 1/S = 437.4$
\newline
$U_p = 0.4 1/S = 447.4$
\newline
$U_p = 0.6 1/S = 458.7$
\newline
$U_p = 0.8 1/S = 474.8$
\newline
$U_p = 1 1/S = 493.1$
\newline
$U_p = 1.2 1/S = 519.5$
\newline
$U_p = 1.4 1/S = 555.6$
\newline
$U_p = 1.6 1/S = 607.9$
\newline
$U_p = 1.8 1/S = 695.0$
\newline
$U_p = 2 1/S = 888.1$
\newline

\end{document}