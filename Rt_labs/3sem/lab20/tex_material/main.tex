\documentclass[a4paper, 14pt]{extarticle}%тип документа
\date{}

\usepackage{indentfirst}


\usepackage{graphicx}
\usepackage{cmap}
\usepackage[T2A]{fontenc}
\usepackage[utf8]{inputenc}

%%\renewcommand{\footrulewidth}{ .0em }
\usepackage[english,russian]{babel}
\usepackage{multirow} % Слияние строк в таблице
\newcommand
{\un}[1]
{\ensuremath{\text{#1}}}


%Русский язык
\usepackage[T2A]{fontenc} %кодировка
\usepackage[utf8]{inputenc} %кодировка исходного кода
\usepackage[english,russian]{babel} %локализация и переносы

%Таблицы
\usepackage[table,xcdraw]{xcolor}
\usepackage{booktabs}

%Математика
\usepackage{amsmath, amsfonts, amssymb, amsthm, mathtools}

%отступы 
\usepackage[left=2cm,right=2cm,top=2cm,bottom=3cm,bindingoffset=0cm]{geometry}

%Вставка картинок
\usepackage{graphicx}
\usepackage{wrapfig, caption}
\graphicspath{}
\DeclareGraphicsExtensions{.pdf,.png,.jpg, .jpeg}
\newcommand\ECaption[1]{%
     \captionsetup{font=footnotesize}%
     \caption{#1}}

%Таблицы
\usepackage[table,xcdraw]{xcolor}
\usepackage{booktabs}

%Графики
\usepackage{pgfplots}
\pgfplotsset{compat=1.9}
\usepackage[english,russian]{babel}
\usepackage{multirow} % Слияние строк в таблице
\newcommand
{\un}[1]
{\ensuremath{\text{#1}}}

\begin{titlepage}
	\begin{center}
		\large 	Московский физико-технический университет \\
		Физтех-школа радиотехники и компьютерных технологий\\
		\vspace{0.2cm}
		
		\vspace{4.5cm}
		Лабораторная работа № 16 \\ \vspace{0.2cm}
		\LARGE \textbf{Шумы в электронных схемах}
	\end{center}
	\vspace{2.3cm} \large
	
	\begin{center}
		Работу выполнили: \\
		Тяжкороб Ульяна  \\
		Шурыгин Антон \\
		Широкова Ксения \\
		Б01-909
	\end{center}
	
	
	\begin{center} \vspace{40mm}
		г. Долгопрудный \\
		2020 \\
	\end{center}
\end{titlepage}


\begin{document}


\begin{figure}[H]
	\centering
	\includegraphics[width =  0.9\linewidth]{cvk}
	\caption{Связанные контуры.}
\end{figure}

\section{Задание №1. Система с индуктивной связью.}

\subsection{Пункт}

Открываем модель, знакомимся со схемой, осознаем состав подготовленных графиков для режима $AC$.

\begin{figure}[h!]
	\centering
			\includegraphics[width=1.1\linewidth]{1.1.jpg}
			\caption{Задание 1, пункт 1}
	\label{A}
\end{figure}


\subsection{Пункт}

Оставляем плот 2 графиков частотных характеристик. 
\newline
Варьируем параметры контуров:
\newline
$R_{1(2)} = [100k,  \: 900k \:| \: 200k]$, $C_{1(2)} [159.2p, \: 165p \: | \: 1p]$, $C_{1(2)} [159.2p, \: 165p \: | \: -1p]$

\begin{figure}[h!]
	\centering
			\includegraphics[width=1.1\linewidth]{1.2.jpg}
            \caption{Задание 1, пункт 2}
	\label{A}
\end{figure}

\begin{figure}[h!]
	\centering
			\includegraphics[width=1.1\linewidth]{1.2_varR1.jpg}
            \caption{Задание 1, пункт 2, варьирование R1}
	\label{A}
\end{figure}

\begin{figure}[h!]
	\centering
			\includegraphics[width=1.1\linewidth]{1.2_varR2.jpg}
            \caption{Задание 1, пункт 2, варьирование R2}
	\label{A}
\end{figure}


\begin{figure}[h!]
	\centering
			\includegraphics[width=1.1\linewidth]{1.2_varC1.jpg}
            \caption{Задание 1, пункт 2, варьирование $C_1 [159.2p, 165p|1p]$}
	\label{A}
\end{figure}


\begin{figure}[h!]
	\centering
			\includegraphics[width=1.1\linewidth]{1.2_varC2.jpg}
            \caption{Задание 1, пункт 2, варьирование $C2 [159.2p, 165p|1p]$}
	\label{A}
\end{figure}


\begin{figure}[h!]
	\centering
			\includegraphics[width=1.1\linewidth]{1.2_varC1_3.jpg}
            \caption{Задание 1, пункт 2, варьирование $С1 [159.2p,155p|-1p]$}
	\label{A}
\end{figure}


\begin{figure}[h!]
	\centering
			\includegraphics[width=1.1\linewidth]{1.2_varC1_4.jpg}
            \caption{Задание 1, пункт 2, варьирование $С2 [159.2p,155p|-1p]$}
	\label{A}
\end{figure}

\begin{figure}[h!]
	\centering
			\includegraphics[width=1.1\linewidth]{1.2varF1.jpg}
            \caption{Задание 1, пункт 3, варьирование $F [0.2, 1|0.2]$}
	\label{A}
\end{figure}


\begin{figure}[h!]
	\centering
			\includegraphics[width=1.1\linewidth]{1.2varF2.jpg}
            \caption{Задание 1,  пункт 3, варьирование $F [0.2, 1|0.2]$}
	\label{A}
\end{figure}


\subsection{Пункт}
Изучим поведение резонансных кривых и фазовых характеристик при $F = [0.2, 1|0.2]$ и $F = [1, 5|1]$. Измерим границы диапазонов изменения фаз на первом и втором контурах: 

на первом контуре -- от $-77.010^{\circ}$ до $75.155^{\circ}$, на втором контуре -- от $-238.454^{\circ}$ до $60.139^{\circ}$,

а также разность фаз между напряжениями на контурах на частоте $f_0$: $93.711^{\circ}$.

Измерив уровни $u_1(f_0)$, $u_2(f_0)$ при $F = 0.5;~1;~2$, проверим формулы 
\begin{equation}
\label{f1}
u_1(f_0) = \frac{1}{1+F^2},~u_2(f_0)= \frac{F}{1+F^2}
\end{equation}

\begin{table}[H]
	\centering
	\caption{Проверка формулы \eqref{f1}.}
	\begin{tabular}{c|ccc} \toprule
		$F$                         & $1$ & $0.5$ & $2$ \\ \midrule
		$u_1(f_0)_{Mccap}$          & 0.5 & 0.8   & 0.2 \\
		$u_1(f_0)_{\text{формула}}$ & 0.5 & 0.8   & 0.2 \\
		$u_2(f_0)_{Mccap}$          & 0.5 & 0.4   & 0.2 \\
		$u_2(f_0)_{\text{формула}}$ & 0.5 & 0.4   & 0.2 \\ \bottomrule
	\end{tabular}
\end{table}

$\Longrightarrow$ формула \eqref{f1} выполняется.

~

\subsection{Пункт}

Измерим значения $F$, при которых возникает: a) провал на первом контуре: $F = 0.6$, b) провал на втором контуре: $F = 1.1$, c) подъём на фазовой характеристике первого контура: $F = 1.1$.

Измерив частоты пересечения нуля фазовой характеристикой $u_1$ при $F = 5$ ($\nu \hm{=} 976.121k,~1.001M,~1.025M$) и при $F = 10$ ( $\nu = 953.430k,~1.005M,~1.054M$), проверим приближённые ($f_0\pm FF_0$) и уточнённые ($f_0\sqrt{1\pm\frac{F}{Q}}$).

~

\subsection{Пункт}

Оставим только плот 1. При критической связи измерим ширину полосы по уровню -3dB эталонного контура ($\Delta f = 10.273k$) и ширину полосы по уровню -9dB резонансной кривой на втором контуре ($\Delta f = 14.279k$). Убедимся, что их отношение составляет $\sqrt{2}$.


\begin{figure}[h!]
	\centering
			\includegraphics[width=1.1\linewidth]{1.5.jpg}
            \caption{Задание 1,  пункт 5}
	\label{A}
\end{figure}

\begin{figure}[h!]
	\centering
			\includegraphics[width=1.1\linewidth]{1.5varR1.jpg}
            \caption{Задание 1,  пункт 5, варьирование $R_3 [60k, 80k|5k]$}
	\label{A}
\end{figure}


Измерим уровни затухания критической кривой при сдвигах по частоте на декаду $F_0$, то есть на $\pm 10F_0 = \pm 50k$ (затухание -- $-34\frac{dB}{\text{дек}_{F_0}}$). Варьируя сопротивление потерь эталонного контура $R = [60k,~80k|5k]$, выясним, что при добротности $Q = 68.6$ ($R \hm{=} 70k,~\Delta f = 14.557k$) его полоса сравнивается с полосой двухконтурной системы. Измерим затухание, вносимое эталонным контуром с этой добротностью при расстройках на декаду $F_0$ (затухание -- $-16.7\frac{dB}{\text{дека}_{F_0}}$). Оценим выигрыш двухконтурной системы по затуханию: выигрыш $\simeq~\text{2 раза}$.

~

\subsection{Пункт}

Изучим поведение резонансных кривых при $F \hm{=} [0.5,~1|0.1]$. Найдём значение $F \hm{=} [0.65,~0.75|0.05]$, при котором полоса двухконтурной системы по критическому уровню -9dB сравнивается с полосой $10k$ эталонного контура: $F = 0.75$. При этом значении $F$ оценим выигрыш по затуханию при расстройке на декаду $F_0$ двухконтурной системы по сравнению с эталоном: у эталона -- $-19.75\frac{dB}{\text{дек}_{F_0}}$, у двухконтурной системы -- $-36.45\frac{dB}{\text{дек}_{F_0}} \Longrightarrow \text{выигрыш $\simeq$ 2 раза}$.

\begin{figure}[h!]
	\centering
			\includegraphics[width=1.1\linewidth]{1.6varF1.jpg}
            \caption{Задание 1,  пункт 6, варьирование $F [0.5, 1|0.1]$}
	\label{A}
\end{figure}

\begin{figure}[h!]
	\centering
			\includegraphics[width=1.1\linewidth]{1.6varF2.jpg}
            \caption{Задание 1,  пункт 6, варьирование $F [0.65, 0.75|0.05]$}
	\label{A}
\end{figure}

~

\subsection{Пункт}

Измерим значение $F$ из диапазона $F = [2.2,~2.6|0.1]$, при котором полоса двухконтурной системы по критическому уровню -9dB сравнивается с полосой $10k$ эталонного контура.($F = 0.75$). При этом значении $F$ измерим ширину полосы $\Delta \omega$ двухконтурной системы по уровню -9dB ($\Delta \omega = 30.532k$) и уровни затухания при расстройках на декаду $F_0$ (у эталона -- $-23\frac{db}{\text{дек}}$, у двухконтурной системы -- $-19.833\frac{dB}{\text{дек}}$). Варьированием сопротивления эталонного контура $R$ добьёмся совпадения его полосы с полосой двухконтурной системы и измерим уровни затухания, вносимого контуром($-18.278\frac{dB}{\text{дек}}$).


\begin{figure}[h!]
	\centering
			\includegraphics[width=1.1\linewidth]{1.7varF.jpg}
            \caption{Задание 1,  пункт 7, варьирование $F [1, 5 \: | \: 1]$}
	\label{A}1
\end{figure}


\begin{figure}[h!]
	\centering
			\includegraphics[width=1.1\linewidth]{1.7varF2.jpg}
            \caption{Задание 1,  пункт 7, варьирование $F [2.2, 2.6 \: | \: 0.1]$}
	\label{A}
\end{figure}



\subsection{Пункт}
При критической связи $F = 1$ измерим затухания на втором контуре при расстройках на декаду $f_0$. Изучим зависимость уровней затухания от $F = [1,5.5|1.5]$. Занесём результаты в таблицу \ref{t2}.

\begin{figure}[h!]
	\centering
			\includegraphics[width=1.1\linewidth]{1.8.jpg}
            \caption{Задание 1,  пункт 8}
	\label{A}
\end{figure}

\begin{figure}[h!]
	\centering
			\includegraphics[width=1.1\linewidth]{1.8_VarF.jpg}
            \caption{Задание 1,  пункт 8, варьирование $F [1, 5.5|1.5]$}
	\label{A}
\end{figure}

\begin{figure}[h!]
	\centering
			\includegraphics[width=1.1\linewidth]{1.8_VarR.jpg}
            \caption{Задание 1,  пункт 8, варьирование $R = [70k, 70k|1k]$}
	\label{A}
\end{figure}


\begin{table}[H]
	\centering
	\caption{Зависимость уровней затухания от $F$}.
	\label{t2}
	\begin{tabular}{c|cc} \toprule
		& \multicolumn{2}{c}{\begin{tabular}[c]{@{}c@{}}Уровень затухания,\\ $\frac{dB}{\text{дек}}$\end{tabular}} \\ \midrule
		$F$   & \multicolumn{1}{c|}{f = $100k$}                              & f = $10Meg$                              \\ \midrule
		$1$   & \multicolumn{1}{c|}{$-94$}                                   & $-133$                                   \\
		$2.5$ & \multicolumn{1}{c|}{$-85$}                                   & $-126$                                   \\
		$4$   & \multicolumn{1}{c|}{$-82$}                                   & $-122$                                   \\
		$5.5$ & \multicolumn{1}{c|}{$-79$}                                   & $-119$                    \\ \bottomrule              
	\end{tabular}
\end{table} 

\subsection{Пункт}

Снимаем зависимость пиковых значений вещественной и мнимой частей вносимой проводимости при двух различных варьированиях.
\newline
Проверяем формулу $Re(Y) = \frac{F^{2}}{Q\rho}$.

\begin{figure}[h!]
	\centering
			\includegraphics[width=1.1\linewidth]{908_work/9_1.png}
            \caption{Задание 1,  пункт 9, варьирование $F [0.5, 1|0.5]$}
	\label{A}
\end{figure}


\begin{figure}[h!]
	\centering
			\includegraphics[width=1.1\linewidth]{908_work/9_2.png}
            \caption{Задание 1,  пункт 9, варьирование $F [1, 2|1]$}
	\label{A}
\end{figure}


\subsection{Пункт}

В режиме Transient проанализировать пеереходные характеристики до напряжений на первом и втором контурах при $ F  = 1$. 


\begin{figure}[h!]
	\centering
			\includegraphics[width=1.1\linewidth]{908_work/10_2.png}
            \caption{Задание 1,  пункт 10, $F = 1$}
	\label{A}
\end{figure}

Варьируем $F [0.1, 0.1|1]$. 

\begin{figure}[h!]
	\centering
			\includegraphics[width=1.1\linewidth]{908_work/10_3_f1.png}
            \caption{Задание 1,  пункт 10, варьирование $F [0.1, 0.1|1]$}
	\label{A}
\end{figure}


Варьируем $F [2, 2|1]$

\begin{figure}[h!]
	\centering
			\includegraphics[width=1.1\linewidth]{908_work/10_3_f2.png}
            \caption{Задание 1,  пункт 10, варьирование $F [2, 2|1]$}
	\label{A}
\end{figure}

Варьируем  $F  [4, 4|1]$

\begin{figure}[h!]
	\centering
			\includegraphics[width=1.1\linewidth]{908_work/10_3_f4.png}
            \caption{Задание 1,  пункт 10, варьирование $F  [4, 4|1]$}
	\label{A}
\end{figure}

Варьируем $F [8, 8|1]$

\begin{figure}[h!]
	\centering
			\includegraphics[width=1.1\linewidth]{908_work/10_3_f8.png}
            \caption{Задание 1,  пункт 10, варьирование $F [8, 8|1]$}
	\label{A}
\end{figure}

Все расчеты, проверки приведны на скриншотах в данном пункте.

\subsection{Пункт}

Установив диапазон моделирования $[2Meg, 600k]$, исследуем частотные и фазовые характеристики при сильной связи. 
\newline
Измерим частоты $f_{\pm}$ пиков при $F = 50$: $f_{+} = 1.414M$, $f_{-} = 816.23k$.
\newline
Проверяем формулу: $f_{\pm} = \frac{f_0}{\sqrt{1 \pm k}}$

\begin{figure}[h!]
	\centering
			\includegraphics[width=1.1\linewidth]{908_work/11_1.png}
            \caption{Задание 1,  пункт 11, варьирование $F  [10, 70|10]$}
	\label{A}
\end{figure}


\begin{figure}[h!]
	\centering
			\includegraphics[width=1.1\linewidth]{908_work/11_2.png}
            \caption{Задание 1,  пункт 11}
	\label{A}
\end{figure}

\newpage

\section{Задание №2. Система с ёмкостной связью.}

\subsection{}

Измерим диапазоны изменения фазовых характеристик на первом и втором контурах:

\begin{figure}[h!]
	\centering
			\includegraphics[width=1.1\linewidth]{2.1_varF1.jpg}
            \caption{Задание 2,  пункт 1, варьирование $F  [1, 4|1]$}
	\label{A}
\end{figure}


на 1 контуре -- от $90^{\circ}$ до $-90^{\circ}$

на 2 контуре -- от $-90^{\circ}$ до $-450^{\circ}$.

Измерим значения $F$, при которых возникает:
\newline
a) провал на первом контуре ($F = 0.5$).
\newline
b) провал на втором контуре ($F = 1$) c) подъём на фазовой характеристике первого контура ($F = 1$).

Снимем зависимость частоты провала на втором контуре от $F = [2, 4|1]$ (таблица \ref{t3}).

\begin{figure}[h!]
	\centering
			\includegraphics[width=1.1\linewidth]{2.1_varF2.jpg}
            \caption{Задание 2,  пункт 1, варьирование $F  [2, 4|1]$}
	\label{A}
\end{figure}


\begin{table}[H]
	\centering
	\caption{Зависимость частоты провала от $F$. }
	\label{t3}
	\begin{tabular}{c|ccc} \toprule
		$F$               & $2$    & $3$    & $4$    \\ \midrule
		$f_{\text{пров}},~\text{Гц}$ & $990k$ & $985k$ & $980k$ \\ \bottomrule
	\end{tabular}
\end{table}

~

\subsection{Пункт}

Измерим уровни затухания при расстройках на $\pm 50k$:

\begin{figure}[h!]
	\centering
			\includegraphics[width=1.1\linewidth]{2.2_varF.jpg}
            \caption{Задание 2,  пункт 2, варьирование $F  [1,4|1]$}
	\label{A}
\end{figure}

1 контур -- $-17 \frac{dB}{\text{дек}}$

2 контур -- $-35 \frac{dB}{\text{дек}}$.

Перейдём на частотный диапазон $[10Meg,100k]$ и измерим уровни затухания при расстройках на декаду $f_0$:

\begin{figure}[h!]
	\centering
			\includegraphics[width=1.1\linewidth]{2.2_freq_range.jpg}
            \caption{Задание 2,  пункт 2, частотный диапазон $[10Meg, 100k]$}
	\label{A}
\end{figure}

1 контур -- $-56\frac{dB}{\text{дек}}$

2 контур -- $-94\frac{dB}{\text{дек}}(\text{вблизи $100k$})$, $-133\frac{dB}{\text{дек}}(\text{вблизи $10Meg$})$



\subsection{Пункт}

Изучим  переходные ххарактеристики при значениях $F = 0.1$,  $F = 1$,  $F = 2$,   $F = 4$. Убедимся в их сходстве с характеристиками системы с индуктивной связью. 

\begin{figure}[h!]
	\centering
			\includegraphics[width=1.1\linewidth]{2.3_varF1.jpg}
            \caption{Задание 2,  пункт 3, варьирование $F  [0.1, 0.1|1]$}
	\label{A}
\end{figure}


\begin{figure}[h!]
	\centering
			\includegraphics[width=1.1\linewidth]{2.3_varF2.jpg}
            \caption{Задание 2,  пункт 3, варьирование $F  [1, 1|1]$}
	\label{A}
\end{figure}

\begin{figure}[h!]
	\centering
			\includegraphics[width=1.1\linewidth]{2.3_varF3.jpg}
            \caption{Задание 2,  пункт 3, варьирование $F   [2, 2|1]$}
	\label{A}
\end{figure}


\begin{figure}[h!]
	\centering
			\includegraphics[width=1.1\linewidth]{2.3_varF4.jpg}
            \caption{Задание 2,  пункт 3, варьирование $F  [4, 4|1]$}
	\label{A}
\end{figure}

\end{document}